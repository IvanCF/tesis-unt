%%%%%%%%%%%%%%%%%%%%%%%%%%%%%%%%%%%%%%%%%%%%%%%%%%%%%%%%%%%%%%%%%%%%%%%%%%%%%%%%
%
% Template license:
% CC BY-NC-SA 3.0 (http://creativecommons.org/licenses/by-nc-sa/3.0/)
%
%%%%%%%%%%%%%%%%%%%%%%%%%%%%%%%%%%%%%%%%%%%%%%%%%%%%%%%%%%%%%%%%%%%%%%%%%%%%%%%%

%----------------------------------------------------------------------------------------
%	PACKAGES AND OTHER DOCUMENT CONFIGURATIONS
%----------------------------------------------------------------------------------------

\documentclass[
11pt, % The default document font size, options: 10pt, 11pt, 12pt
%oneside, % Two side (alternating margins) for binding by default, uncomment to switch to one side
%chapterinoneline,% Have the chapter title next to the number in one single line
spanish,
singlespacing, % Single line spacing, alternatives: onehalfspacing or doublespacing
%draft, % Uncomment to enable draft mode (no pictures, no links, overfull hboxes indicated)
%nolistspacing, % If the document is onehalfspacing or doublespacing, uncomment this to set spacing in lists to single
%liststotoc, % Uncomment to add the list of figures/tables/etc to the table of contents
%toctotoc, % Uncomment to add the main table of contents to the table of contents
parskip, % Uncomment to add space between paragraphs
%codirector, % Uncomment to add a codirector to the title page
headsepline, % Uncomment to get a line under the header
]{MastersDoctoralThesis} % The class file specifying the document structure

% mis paquetes adicionales
\usepackage[utf8]{inputenc}
\usepackage{pdflscape} % paquete para girar la hoja
\usepackage{colortbl, xcolor}
\usepackage{pdfpages}
%----------------------------------------------------------------------------------------
%	INFORMACIÓN DE LA MEMORIA
%----------------------------------------------------------------------------------------
%--inicio portada

\thesistitle{Desarrollo de un sistema BMS - IoT \\ para el control automatizado del consumo eléctrico basado en el protocolo  \\MQTT} % El títulos de la memoria, se usa en la carátula y se puede usar el cualquier lugar del documento con el comando \ttitle

% Nombre del posgrado, se usa en la carátula y se puede usar el cualquier lugar del documento con el comando \degreename
%\posgrado{Carrera de Especialización en Sistemas Embebidos} 
%\posgrado{UNIVERSIDAD NACIONAL DE TRUJILLO} 
%\posgrado{ESCUELA PROFESIONAL DE INFORMÁTICA}
%\posgrado{Maestría en Sistemas Embebidos} 
%\posgrado{Maestría en Internet de las cosas}

\author{B.Sc. Daniel Iván Cruz Flores}
\authordos{B.Sc. Leoncio Moya Carranza}
 % Tu nombre, se usa en la carátula y se puede usar el cualquier lugar del documento con el comando \authorname

\director{Dra. Patricia Pereyra Salvador} % El nombre del director, se usa en la carátula y se puede usar el cualquier lugar del documento con el comando \dirname
%\codirector{Nombre del codirector (pertenencia)} % El nombre del codirector si lo hubiera, se usa en la carátula y se puede usar el cualquier lugar del documento con el comando \codirname.  Para activar este campo se debe descomentar la opción "codirector" en el comando \documentclass, línea 23.

%\juradoUNO{Mg. Ing. Martín Menendez (FIUBA)} % Nombre y pertenencia del un jurado se usa en la carátula y se puede usar el cualquier lugar del documento con el comando \jur1name
%\juradoDOS{Mg. Ing. Christian Marcelo Yanez Flores (FIUBA)} % Nombre y pertenencia del un jurado se usa en la carátula y se puede usar el cualquier lugar del documento con el comando \jur2name
%\juradoTRES{Esp. Ing. Santiago Salamandri (FIUBA)} % Nombre y pertenencia del un jurado se usa en la carátula y se puede usar el cualquier lugar del documento con el comando \jur3name

%\ciudad{Ciudad Autónoma de Buenos Aires}
%\ciudad{ciudad de Buenos Aires}

%\fechaINICIO{agosto de 2020}
%\fechaFINAL{junio de 2022}


\keywords{Sistemas embebidos,UNT} % Keywords for your thesis, print it elsewhere with \keywordnames
%--fin portada
\begin{document}


\frontmatter % Use roman page numbering style (i, ii, iii, iv...) for the pre-content pages

\pagestyle{plain} % Default to the plain heading style until the thesis style is called for the body content






%----------------------------------------------------------------------------------------
%	CONTENIDO DE LA MEMORIA  - DEDICATORIA
%----------------------------------------------------------------------------------------

\dedicatory{\textbf{Dedico esta tesis:\vspace{1.0cm}}

A Dios por ser mi guía y acompañarme en el transcurso de mi vida, brindándome paciencia y sabiduría para culminar con éxito mis metas propuestas.

A mis padres, a mi esposa Gabriela y a mi hijo Carlos Daniel por ser mis principales pilares de vida y por haberme apoyado incondicionalmente para no decaer cuando todo parecía complicado e imposible.

A mis hermanos por estar siempre presentes, por el apoyo y el soporte moral.

A mi tía Corina Cruz Perez por todo su apoyo incondicional para culminar con éxito esta meta.

\vspace{1.0cm}
\hspace{8cm} Daniel Iván Cruz Flores

\vspace{1.2cm}

...Sección de la dedicatoria pendiente del autor 2....

\vspace{1.0cm}
\hspace{8cm} Leoncio Moya Carranza
}


 % escribir acá si se desea una dedicatoria
%----------------------------------------------------------------------------------------
%	CONTENIDO DE LA MEMORIA  - AGRADECIMIENTOS
%----------------------------------------------------------------------------------------

\begin{acknowledgements}
%\addchaptertocentry{\acknowledgementname} % Descomentando esta línea se puede agregar los agradecimientos al índice
\vspace{1.5cm}

A todos los docentes que con su conocimiento nos motivaron al desarrollo personal y profesional.

A nuestra asesora Dra. Patricia Pereyra Salvador Torres por su buena predisposición y orientación durante la realización del presente trabajo.


\end{acknowledgements}

% incluir documento del acta de sustentacion 
\includepdf[pages=1]{ActaSustentacion.pdf}
%----------------------------------------------------------------------------------------
%	RESUMEN - ABSTRACT 
%----------------------------------------------------------------------------------------

\begin{abstract}
\addchaptertocentry{\abstractname} % Add the abstract to the table of contents
%
%The Thesis Abstract is written here (and usually kept to just this page). The page is kept centered vertically so can expand into the blank space above the title too\ldots
\centering

Esta memoria describe el diseño e implementación de un sistema automatizado para el control y monitoreo de variables de consumo energético en el hogar. El sistema se destaca especialmente por unificar resultados de una red de sensores por vía inalámbrica usando el protocolo MQTT. Asimismo, incluye el desarrollo de una plataforma web para visualizar resultados, un módulo de medición de temperatura, un módulo actuador, un módulo de consumo de corriente eléctrica y un servidor local.
			
En este trabajo se aplicaron conocimientos de protocolos de Internet, redes inalámbricas, base de datos relacional, desarrollo de aplicaciones web, programación orientada a objetos, ciberseguridad y testing de software.	
				
\end{abstract}

%----------------------------------------------------------------------------------------
%	LISTA DE CONTENIDOS/FIGURAS/TABLAS
%----------------------------------------------------------------------------------------

\listoffigures % Prints the list of figures

\listoftables % Prints the list of tables

\tableofcontents % Prints the main table of contents



%----------------------------------------------------------------------------------------
%	CONTENIDO DE LA MEMORIA  - CAPÍTULOS
%----------------------------------------------------------------------------------------

\mainmatter % Begin numeric (1,2,3...) page numbering

\pagestyle{thesis} % Return the page headers back to the "thesis" style

% Incluir los capítulos como archivos separados desde la carpeta Chapters

% Chapter 1

\chapter{Introducción general} % Main chapter title

\label{Chapter1} % For referencing the chapter elsewhere, use \ref{Chapter1} 
\label{IntroGeneral}
La tecnología de automatización orientada a viviendas y lugares de trabajo está tomando cada vez mayor relevancia en nuestras vidas, aunque existe una gran variedad de soluciones que ofrecen dotar de tecnología a ciertos elementos en un ambiente doméstico, su principal desventaja es que casi siempre la recolección de datos se realiza con sensores que dependen de una conexión a Internet para enviar sus valores a un servidor central para su gestión. Esta forma de trabajo permite la facilidad de uso pero con la desventaja de que la gestión funciona solo si existe conexión a Internet, además los sensores más comerciales del mercado actual están destinados a un uso independiente con su propia aplicación móvil, complicando la necesidad de tener una única solución centralizada ante el uso de múltiples sensores.

El presente proyecto se destaca especialmente por centralizar y unificar resultados de la red de sensores en un sistema web principal de monitoreo y control sin depender de la conexión a Internet para su funcionamiento. Para lograr esta tarea se trabajará con la recolección de datos de sensores ubicados en distintos puntos de estudio de una vivienda o ambiente conectados a una red local dedicada. Cada una de las lecturas de los sensores serán enviadas a una unidad central local mediante el protocolo MQTT por vía inalámbrica, dejando la dependencia del Internet solo para los accesos remotos.

El módulo central local contará con la función para replicar los valores recolectados hacia un servicio en la nube, tarea que se realizará mientras tenga conexión a Internet.

Las funciones principales a implementar están orientados a confort (control de ventiladores y luces) y consumo energético de electrodomésticos.  

El sistema BMS a desarrollar contará de dos formas de acceso: la primera es vía red local y la segunda vía Internet desde cualquier dispositivo mediante un navegador.
%----------------------------------------------------------------------------------------

% Define some commands to keep the formatting separated from the content 
\newcommand{\keyword}[1]{\textbf{#1}}
\newcommand{\tabhead}[1]{\textbf{#1}}
\newcommand{\code}[1]{\texttt{#1}}
\newcommand{\file}[1]{\texttt{\bfseries#1}}
\newcommand{\option}[1]{\texttt{\itshape#1}}
\newcommand{\grados}{$^{\circ}$}

%----------------------------------------------------------------------------------------

%\section{Introducción}

%----------------------------------------------------------------------------------------

%\section{Conceptos generales}

%En esta sección se describen aspectos esenciales para poder conocer y entender las tecnologías %y servicios usados en el trabajo.
\vspace{5.0cm}

\section{Justificación de la investigación}

Las necesidades de crear sistemas autómatas que permitan el control, seguridad, ahorro de energía y comunicación, son prestaciones que aportan valor añadido a la gestión técnica en los proyectos de edificaciones para los sectores público y privado creando entornos y ambientes que facilitan la habitabilidad y uso de una infraestructura moderna que aprovecha y hace uso de los diversos avances tecnológicos.

El desarrollo sostenible del país, ha permitido que en las últimas décadas los sectores productivos, tales como la industria, la construcción, la minería, entre otros, experimenten un auge y crecimiento de proyectos de inversión pública y privada. Para lo cual se genera la necesidad de contar con profesionales altamente capaces de aprovechar y desarrollar los diversos medios, elementos y avances tecnológicos en beneficio de la sociedad y los sectores productivos del país. \citep{ARTICLE:2}.

Por lo mencionado en los párrafos anteriores nuestra labor como profesionales con bases tecnológicas y científicas es crear los medios que permita acercar la tecnología a las personas, brindándoles un abanico de soluciones cada vez más amigables e informativas para que puedan conocer y usar. Por tanto, este proyecto busca ser una alternativa para aquellas personas que buscan una solución de automatización inteligente para sus viviendas u oficinas, así como brindarles la capacidad de interconexión de sus dispositivos por medio de nuestra plataforma.

\section{Formulación del problema}

¿Cómo mejorar la gestión y control del consumo eléctrico mediante un sistema BMS basado en los principios de IoT?

\section{Hipótesis}

El desarrollo de un sistema BMS - IoT para automatizar la gestión eléctrica permitirá reducir el error humano en la toma de muestras y cálculo del consumo eléctrico mensual dentro de un edificio habitacional, de oficinas u otro entorno. Reduciendo el tiempo de muestreo que esta tarea conlleva y aumentando la disponibilidad de los datos de consumo para su acceso en tiempo real.

\section{Objetivos}

El propósito de este trabajo es diseñar y desarrollar un sistema prototipo operativo funcional capaz de controlar y monitorear el consumo eléctrico mediante el protocolo MQTT y para lograrlo se plantea los siguientes objetivos.

\subsection{Generales}
\begin{itemize}
\item Diseñar y desarrollar un sistema BMS - IoT para el la gestión y control 
automatizado del consumo eléctrico utilizando el protocolo MQTT.
\end{itemize}
\subsection{Específicos}

\begin{itemize}
\item Diseñar una red IoT para un sistema BMS considerando las variables existentes en los edificios habitacionales.
\item Diseñar e Implementar la interfaz de comunicación y toma de datos de los sensores de forma simultanea mediante comunicación inalámbrica con el protocolo MQTT.
\item Diseñar e Implementar un módulo integrador - replicador de comunicación entre la red interna y la comunicación en la nube.
\item Diseñar e Implementar los módulos electrónicos con sensores y actuadores para medir el consumo eléctrico y temperatura ambiente.
\item Diseñar e implementar un software web responsivo a medida para el sistema BMS - IoT.
\end{itemize}

%-----------------------------------------------------------------------



%----------------------------------------------------------------------------------------
%\section{Alcance}


%\subsection{Alcance}

%El desarrollo de trabajo incluye:

%\begin{itemize}
%\item Diseño e implementación de un módulo principal local.
%\item Diseño e implementación del módulo replicador de datos local-nube.
%\item Diseño e implementación del módulo actuador y de consumo eléctrico.
%\item Diseño e implementación módulo de control de temperatura.
%\item Diseño e implementación de un \emph{software} web BMS a medida.
%\end{itemize}

%\subsection{Objetivos}
%\begin{itemize}
%\item Diseñar y desarrollar un sistema IoT para medir el consumo eléctrico.
%\item Diseñar y desarrollar un sistema que no dependa de una conexión a Internet para su funcionamiento.
%\item Diseñar y desarrollar un \emph{software} a medida para la gestión de control y monitoreo de una vivienda o edificio.
%\item Diseñar e implementar módulos con comunicación Wi-Fi para el control y monitoreo de energía eléctrica.
%\end{itemize}

%----------------------------------------------------------------------------------------
\let\cleardoublepage\clearpage % para eliminar pagina en blanco siguiente
\chapter{Introducción específica} % Main chapter title

\label{Chapter2}

%----------------------------------------------------------------------------------------
%	SECTION 1
%----------------------------------------------------------------------------------------
%En este capítulo se presentan las tecnologías utilizadas e incorporadas en este trabajo. 

En este capítulo se explica el funcionamiento general del sistema implementado y se brinda una introducción a las diferentes tecnologías utilizadas en este trabajo.

\section{Funcionamiento general del sistema}

Este trabajo presenta una propuesta de solución a partir del desarrollo de un prototipo mínimo viable de un sistema IoT para integrar, centralizar y unificar resultados de una red de sensores y actuadores, mediante un sistema web de monitoreo y control, así como la construcción de módulos propios para dicha tarea, sin la necesidad de una conexión a Internet para su funcionamiento. 

Todos estos componentes trabajan en conjunto para brindar al cliente una solución tecnológica que sirva como herramienta visualmente amigable y que pueda ser el soporte en las tareas de gestión de consumo eléctrico. Para lograrlo, se desarrollaron módulos que integran \emph{software} y \emph{hardware}, estos hacen posible la adquisición de datos de sensores y actuadores ubicados en distintos puntos de una vivienda, conectados a una red local vía Wi-Fi. Cada lectura de los sensores permite el envío de datos a un servidor central local mediante el protocolo MQTT por vía inalámbrica, siendo el módulo principal local el responsable de registrar los valores de los sensores ante cualquier corte de Internet. La figura \ref{fig:diagrama1} ilustra el diagrama de componentes del sistema y la lógica de conexión.

\begin{figure}[htbp]
	\centering
	\includegraphics[width=0.9\textwidth]{./Figures/bloques.png}
	\caption{Diagrama de bloques del sistema.}

	\label{fig:diagrama1}
\end{figure}

El sistema tiene la capacidad de permitir el acceso por medio de cualquier dispositivo que cuente con un navegador web y con conectividad a Internet a cualquier usuario, ya sea desde dentro de la red local o desde fuera. Para alcanzar esta funcionalidad, fue necesario desarrollar un módulo de \emph{software} que permita la replicación de datos desde la red local hacia un broker remoto ubicado en la nube. La necesidad de replicación de los datos hacia la nube solo se da mientras exista conexión a Internet. Para los casos de corte de Internet, el módulo replicador enviará los datos de forma automática la próxima vez que detecte el servicio.% de conexión a Internet. 

Los componentes claves dentro de la solución propuesta y que se integran en la arquitectura IoT usada en el proceso de desarrollo son: 


\begin{itemize}
\item Dispositivos IoT: son los módulos diseñados a partir de la integración de \emph{software}, \emph{firmware} y \emph{hardware}; es posible conectarlos de forma inalámbrica a una red más amplia.
\item Redes: los routers domésticos, puntos de acceso y las configuraciones de las redes o las puertas de enlaces son los responsables de conectar varios dispositivos IoT a la nube.
\item Servidor local: es aquel servidor instalado con el fin de trabajar \emph{offline} y \emph{online}. Es una alternativa útil cuando existe corte del servicio de Internet. El servidor local es el módulo que contiene el broker MQTT. 
\item Nube: servidores remotos en centros de datos que consolidan y almacenan la información con seguridad. Son servicios utilizados en el trabajo para garantizar el acceso remoto de usuarios al sistema.
\end{itemize}

\section{Servicios en la nube}

El \emph{Cloud Computing} o servicios en la nube cobra cada vez más relevancia en las empresas debido, principalmente, a la ventaja de no tener que hacer grandes inversiones en infraestructuras que mantengan aplicaciones, plataformas o servidores propios.

Los servicios en la nube se clasifican en:

\begin{itemize}
\item Infraestructura como servicio (IaaS): esta categoría ofrece servicios de infraestructura, entre ellos está la distribución de recursos de computación y almacenamiento cuyos precios varían de acuerdo al consumo realizado. Las empresas que los contratan nunca ven el equipo físico, pero sí pueden tener acceso a ellos al momento de usar el servicio deseado \citep{BOOK:2}.

%Ejemplo de IaaS:

%\begin{itemize}
%\item Amazon Web Services
%\item Microsoft Azure
%\item Google Cloud Platform
%\end{itemize}

%\vspace{0.5cm}

\item Plataforma como servicio (PaaS): este servicio ofrece plataformas de desarrollo sin necesidad de adquirir tecnología de costo muy elevado. El \emph{hardware} y el \emph{software} en este modelo es administrado por el proveedor del servicio, además de que los desarrolladores no se preocupan por el rendimiento del hardware ni por las actualizaciones del sistema, ya que todo lo realiza el proveedor \citep{BOOK:2}.
 
%Ejemplos de PaaS:

%\begin{itemize}
%\item AWS Elastic Beanstalk
%\item Azure App Service
%item Google App Engine

%\end{itemize}

%\vspace{0.5cm}


\item \emph{Software} como servicio (SaaS): constituye el modelo más utilizado porque, además de brindar servicio de \emph{software}, ofrece también el almacenamiento de la información. Este modelo permite simplicidad de integración y escalabilidad \citep{BOOK:2}. 

%Ejemplos de SaaS:

%\begin{itemize}
%\item Microsoft Office 365
%\item Aplicaciones web de Google
%\end{itemize}

%\vspace{0.5cm}

\end{itemize}

En la figura \ref{fig:servicios} se muestra una representación gráfica para diferenciar las capas y su orientación para cada modelo de servicio.



%\vspace{1.5cm}

\begin{figure}[htbp]
	\centering
	\includegraphics[width=0.8\textwidth]{./Figures/servicios.png}
	\caption{Tipos de servicio y orientación por rol \protect\footnotemark.}

	\label{fig:servicios}
\end{figure}

\footnotetext{Imagen tomada de \url{https://openwebinars.net/blog/tipos-de-cloud-computing/}}

%\vspace{1cm}

Para el trabajo se usó el servicio tipo PaaS en la creación y configuración del broker remoto así como para el almacenamiento de la aplicación web y para la gestión de la base de datos.


%Para el trabajo se consideró el servicio de \emph{cloud computing} tipo PaaS, porque permite la creación y configuración del broker remoto, el servidor Apache para la aplicación web y el gestor de base de datos MySQL.

El plan utilizado se divide en dos categorías, la primera está definida por el servicio del servidor web para alojar la aplicación web de monitoreo y control (réplica) y la segunda por el servicio del broker remoto que permite la comunicación directa con el broker local. La comunicación entre ambos servicios se hace mediante el protocolo MQTT utilizando la biblioteca  \emph{Eclipse Paho JavaScript Client}, se puede encontrar mayor información en su página oficial \citep{WEBSITE:41}. 

Las principales características  del servicio contratado para la implementación del sistema de monitoreo se muestran en la tabla \ref{tab:serverweb} y las características del servicio para el broker remoto se muestran en la tabla  \ref{tab:brokerremoto}.


\begin{table}[h]
	\centering
	\caption[Características del servicio en la nube]{Características del servicio en la nube.}
	\begin{tabular}{p{7cm} p{5cm} }    
		\toprule
		\textbf{Característica} 	 & \textbf{Detalle}  \\
		\midrule
		Sistema operativo  & GNU/Linux Centos\\		
		Espacio de almacenamiento & 1000 MB \\
		Transferencia mensual  & 10 GB\\				
		Cantidad base de datos 	  & ilimitados\\
		Acceso FTP 	  & sí\\
		Backup diario y semanal 	  & sí\\
		Soporte 24/7 	  & sí\\
		Seguridad - Firewall	  & sí\\
		Certificados SSL/TLS	  & sí\\
		PHP	  & V7 y V8\\
		MySQL y	phpMyAdmin  & sí\\
		Cron Jobs	  & sí\\
		\bottomrule
		\hline
	\end{tabular}
	\label{tab:serverweb}
\end{table}

%%%%%%%%%%%%%%%%%%%%%%%%%%%%%%%%%%%%%%%%%%%%%%%%%%%%%

\begin{table}[h]
	\centering
	\caption[Características del broker remoto]{Características del broker remoto.}
	\begin{tabular}{p{5cm} p{7cm} }    
		\toprule
		\textbf{Característica} 	 & \textbf{Detalle}  \\
		\midrule
		Conexiones activas  & 200\\		
		Mensajes por segundo & 10 k \\
		Interface  & MQTT interface, HTTP interface\\		
		Compatibilidad & arduino, javaScript, processing, ruby \\		
		Deployment 	  & por instancias\\
		Envíos y recepción de datos & objetos codificados JSON\\
		
		\bottomrule
		\hline
	\end{tabular}
	\label{tab:brokerremoto}
\end{table}




%En la figura \ref{fig:capas-servicios} se muestran las capas de cada servicio descrito y el acceso que el cliente tiene con cada modelo (capas de color verde).

%\begin{figure}[htbp]
%	\centering
%	\includegraphics[width=.8\textwidth]{./Figures/capas-servicios.png}
%	\caption{Infraestructura por capas según el servicio  \citep{BOOK:2}.}

%	\label{fig:capas-servicios}
%\end{figure}

%\footnotetext{Imagen tomada del libro \textit{Cloud Computing for Pymes}. \citep{BOOK:1}}}



\section{Protocolo MQTT}

MQTT  (\textit{Message Queue Telemetry Transport}) es un protocolo de red ligero de mensajería estándar para IoT. Está diseñado como un transporte de mensajería extremadamente liviano y se utiliza en una amplia variedad de industrias, como la automotriz, las telecomunicaciones, el petróleo y el gas, entre otros  \citep{WEBSITE:4}. 

Para construir una red usando MQTT es necesario entender los conceptos que se utilizan para crear una red IoT: 

\begin{itemize}
\item El broker MQTT: el servidor o broker es el programa que se encarga de recepcionar los mensajes enviados por los clientes y distribuirlos entre sí en un sistema publicador-suscriptor. Los clientes envían periódicamente paquetes y esperan la respuesta del broker, como se ilustra con la figura \ref{fig:broker}. 

El broker MQTT usado en el trabajo es Eclipse Mosquitto, por ser de código abierto.

\begin{figure}[htbp]
	\centering
	\includegraphics[width=.75\textwidth]{./Figures/broker.jpg}
	\caption{Funcionamiento del broker MQTT \protect\footnotemark..}
	\label{fig:broker}
\end{figure}
\footnotetext{Imagen tomada de \url{https://www.factor.mx/portal/base-de-conocimiento/mqtt/}}

%\vspace{1cm}
\item Comunicación MQTT: es la función de transporte de mensajes entre dispositivos IoT. El protocolo generalmente se ejecuta sobre TCP / IP ; sin embargo, cualquier protocolo de red que proporcione conexiones bidireccionales ordenadas y sin pérdidas puede admitir MQTT. Está diseñado para conexiones con ubicaciones remotas donde existen restricciones de recursos o el ancho de banda de la red es limitado \citep{WEBSITE:3}. Su modelo de comunicación se puede ver en la figura \ref{fig:mqtt}.

%\vspace{0.5cm}



La comunicación MQTT puede estar cifrada mediante TLS (\emph{Transport Layer Security}) y contar con credenciales de acceso para el control de los canales de envío y recepción. Al broker se le puede conectar un sinfín de dispositivos como teléfonos móviles, computadoras, sensores, actuadores, lámparas, relojes, bombas de agua, refrigeradores, cocinas y mucho más. 

\begin{figure}[htbp]
	\centering
	\includegraphics[width=0.95\textwidth]{./Figures/mqtt.png}
	\caption{Ejemplo de funcionamiento del protocolo MQTT \protect\footnotemark.}
	\label{fig:mqtt}
\end{figure}

\footnotetext{Imagen tomada de \url{https://mqtt.org/}}

\end{itemize}

%\section{Elementos del broker MQTT}


%\begin{itemize}
%\item Cliente: un dispositivo que puede publicar mensajes, suscribirse para recibir mensajes, o ambos.
%\item Broker: es el servidor que acepta mensajes publicados por clientes y los difunde entre los clientes suscritos.
%\item Publicar: cuando un cliente envía un mensaje al broker usando un tópico.
%\item Tópico: los mensajes deben estar etiquetados con algún tópico o tema. Los clientes se suscriben a tópicos específicos, de manera que solo reciben los mensajes publicados con dichos tópicos. 
%\end{itemize}


\section{Eclipse Mosquitto} 
Es un agente de mensajes de código abierto (con licencia EPL / EDL) cuya versión 2.0.14 implementa las versiones 5.0, 3.1.1 y 3.1 del protocolo MQTT. Mosquitto es liviano y adecuado para su uso en todos los dispositivos, desde computadoras de placa única de baja potencia hasta servidores completos \citep{WEBSITE:5}.

%El protocolo MQTT proporciona un método ligero para realizar mensajes mediante un modelo de publicación / suscripción. Esto lo hace adecuado para la mensajería de Internet de las cosas, como con sensores de baja potencia o dispositivos móviles como teléfonos, computadoras integradas o microcontroladores \citep{WEBSITE:5}.

Mosquitto es parte de la Fundación Eclipse \citep{WEBSITE:5}, es un proyecto de IoT \citep{WEBSITE:39} y está patrocinado por la compañía Cedalo \citep{WEBSITE:40}. 

\section{Componentes del módulo principal} 

El hardware del módulo principal integra la placa raspberry Pi 4 modelo B de 8 GB como placa base. La Raspberry Pi es una serie de ordenador de placa reducida, ordenador de placa única u ordenador de placa simple (SBC) de bajo costo desarrollado en el Reino Unido por la Raspberry Pi Foundation \citep{WEBSITE:6}.

La placa Raspberry Pi 4 es una pequeña computadora de escritorio de doble pantalla con opciones de salida en 4K, se puede usar como cerebro de robot, centro de hogar inteligente, centro multimedia, núcleo de IA (inteligencia artificial) en red y mucho más. La figura \ref{fig:rpi4} muestra sus principales especificaciones.

Las especificaciones técnicas de la computadora Raspberry Pi 4 son \citep{WEBSITE:7}:

\begin{itemize}
\item Broadcom BCM2711, SoC de 64 bits Cortex-A72 de cuatro núcleos (ARM V8) a 1,5 GHz.
\item SDRAM LPDDR4-3200 de 2 GB, 4 GB u 8 GB (según el modelo).
\item 2,4 GHz y 5,0 GHz IEEE 802.11ac inalámbrica, Bluetooth 5.0.
\item Gigabit Ethernet, 2 puertos USB 3.0 y 2 puertos USB 2.0.
\item Cabecera GPIO estándar Raspberry Pi de 40 pines (totalmente compatible con las placas anteriores).
\item Puertos micro-HDMI (hasta 4kp 60 compatible).
\item Ranura para tarjeta microSD para cargar el sistema operativo y el almacenamiento de datos.
\item 5 VCC a través del conector USB-C (mínimo 3 A).
\item 5 VCC a través del encabezado GPIO (mínimo 3 A).
\item Power over Ethernet (PoE) habilitado (requiere PoE HAT separado).
\item Temperatura de funcionamiento ambiente: 0 °C - 50 °C.
\end{itemize}

%%%%%%%%%%%%%%%%%%%%%%%%%%%%%%%%%%%%%%%%%%%%%%%%%%%%%%%%%%%%%%%%%%%
\begin{figure}[htbp]
	\centering
	\includegraphics[width=1.0\textwidth]{./Figures/rpi4.png}
	\caption{Computadora Raspberry Pi 4 \protect\footnotemark. \citep{WEBSITE:7}}

	\label{fig:rpi4}
\end{figure}

\footnotetext{Imagen de \url{https://www.raspberrypi.com/products/raspberry-pi-4-model-b/specifications/}}

%\vspace{1cm}

%%%%%%%%%%%%%%%%%%%%%%%%%%%%%%%%%%%%%%%%%%%%%%%%
\subsection{Alimentación, gabinete y unidad de almacenamiento}

La fuente de alimentación para la Raspberry Pi 4 es de 3 A como se aprecia en la figura \ref{fig:placarpi4}. La unidad de almacenamiento contiene una unidad de memoria extraíble microSD de 64 GB y de clase 10 para garantizar alta velocidad de lectura y escritura durante el procesamiento.
% La figura \ref{fig:microsd} muestra la microSD del módulo.

\begin{figure}[htpb]
\centering 
\includegraphics[width=0.8\textwidth]{./Figures/placa.png}
\caption{Mainboard y fuente de alimentación del servidor.}
\label{fig:placarpi4}
\end{figure}
%\begin{figure}[htpb]
%\centering 
%\includegraphics[width=0.7\textwidth]{./Figures/card.png}
%\caption{Tarjeta de almacenamiento del servidor local.}
%\label{fig:microsd}
%\end{figure}
El case Argon One Pi4 V2 es el gabinete usado para este trabajo como componente para la integración de elementos del módulo principal. La figura \ref{fig:armado} muestra las partes del gabinete.  
\vspace{0.5cm}

%Características del gabinete \citep{WEBSITE:16}:

%\begin{itemize}
%\item Tiene una placa Raspberry Pi Sata, que está diseñada para maximizar las transferencias de datos de alta velocidad para el uso de SSD Raspberry Pi 4; lo que la hace perfecta para el uso multimedia y otras aplicaciones. 
%\item La funda SSD Raspberry Pi solo es compatible con SSD M.2 SATA con llave B-Key o B+M.
%\item El SSD se conecta al Raspberry Pi 4 a través del puente USB en un puerto USB 3.0. 
%\item El case Argon incluye dos puertos HDMI de tamaño completo e IR integrado para la funcionalidad remota y también posee arranque automático. 
%\item El case Argon Raspberry Pi 4 tiene todos los pines GPIO accesibles en la parte superior de la funda mientras están protegidos por una cubierta magnética extraíble cuando no están en uso. 
%\item El case Argon ONE Raspberry Pi 4 obtiene la mejor experiencia de refrigeración gracias al \emph{software} Pi Fan \citep{WEBSITE:42} mientras que la funda actúa como un disipador de calor conectado a la CPU.
%\end{itemize}


\begin{figure}[htpb]
\centering 
\includegraphics[width=1.0\textwidth]{./Figures/argon2.jpg}
\caption{Partes del gabinete Argon One Pi4 V2 \protect\footnotemark.}
\label{fig:armado}
\end{figure}
\footnotetext{Imagen tomada de \url{https://www.dfrobot.com/product-2090.html}}

\subsection{Sistema operativo para el servidor local}

En la actualidad existen mucha variedad de sistemas operativos para la placa Raspberry Pi, pero para este trabajo se usó el sistema operativo oficial y recomendado por la Raspberry Pi Foundation, llamado ``Raspberry Pi OS'' - version 11 basado en el kernel 5.15.

%La web oficial ofrece diversas versiones a las cuales se puede acceder con descarga directa como se ilustra con la figura \ref{fig:so}.

%\begin{figure}[htbp]
%	\centering
%	\includegraphics[width=1.0\textwidth]{./Figures/so.png}
%	\caption{Versiones del sistema operativo para Raspberry Pi \protect\footnotemark.}
%	\label{fig:so}
%\end{figure}
%\footnotetext{Imagen tomada de \url{https://www.raspberrypi.com/software/operating-systems/}}

\section{Hardware, sensores y actuadores}

Los componentes principales usados en el desarrollo de cada módulo están formados por una placa base NodeMCU, sensores y actuadores.

\subsection{Placa NodeMCU ESP8266}

La tarjeta NodeMCU es de bajo costo y está basada en el procesador ESP8266, que es muy utilizado en la realización de proyectos IoT, ya que dispone de Wi-Fi integrado. El procesador se programa en Lua \citep{WEBSITE:38}, pero también se puede programar con C/C++. Su principal característica es que trabaja a 3,3 V.  Además, ofrece más ventajas como la incorporación de un regulador de tensión integrado, así como un puerto USB de programación. 

%En el mercado actual se encuentran dos versiones muy utilizadas de la familia NodeMCU y la forma rápida de diferenciar la V2 de la V3, es fijarnos en el conversor serial que monta y en su tamaño. El CP2102 (V2) que es cuadrado, y el CH340G (V3) que es más alargado respectivamente,.
 
Para el trabajo se usó la placa versión 3 del NodeMCU ESP8266 que contiene el driver CH340G requerido para operar el circuito integrado con la interfaz USB. La placa se ilustra con la figura \ref{fig:nodemcu}:

\begin{figure}[htbp]
	\centering
	\includegraphics[width=.8\textwidth]{./Figures/nodemcuV3.jpg}
	\caption{Modelo y dimensiones de la placa NodeMCU ESP8266 V3.}

	\label{fig:nodemcu}
\end{figure}

\vspace{1.5 cm}
Las especificaciones técnicas del NodeMCU son:

\begin{itemize}
\item Utiliza chip CH340G (USB).
\item Tensión de alimentación: 4,5 V~9 V (10 V max) y/o alimentación por USB.
\item Tensión de pines I/O: 3,3 V.
\item Wireless 802.11 b/g/n standard.
\item Wi-Fi a 2,4 GHz, soporta encriptación WPA/WPA2.
\item Soporta tres modos de operación: STA/AP/STA+AP.
\item Pila de almacenamiento para protocolo TCP/IP (5 conexiones máximo).
\item Pines: D0~D8, SD1~SD3 pueden ser usados como GPIO, PWM, IIC con capacidad de drenar 15 mA por pin.
\item 1 canal ADC: AD0.
\item Consumo de corriente continua ≈ 70 mA (200 mA MAX), Standby: <200 uA.
\item Velocidad de transmisión: 110 - 460800 bps.
\item Soporta interfaz de comunicación UART/GPIO.
\item OTA: \emph{Remote firmware upgrade}.
\item Soporta \emph{Smart Link Smart Networking}.
\item Temperatura de trabajo: -40 ℃ - +125 ℃.
\item Memoria: 4 MByte.
\end{itemize}

\subsection{Sensor de temperatura y humedad DHT11}

El DHT11 es un sensor de humedad relativa y temperatura de media precisión de bajo costo. La salida suministrada es de tipo digital y utiliza solamente un pin de datos. Es usado en aplicaciones académicas relacionadas al control automático de temperatura, aire acondicionado, monitoreo ambiental en agricultura y más. El sensor se muestra en la figura \ref{fig:dht11}.

Utilizar el sensor DHT11 con las plataformas Arduino, Raspberry Pi y NodeMCU es muy sencillo tanto a nivel de \emph{firmware} como \emph{hardware}. A nivel de \emph{software} se dispone de bibliotecas para Arduino con soporte para el protocolo \emph{single bus}. En cuanto al hardware, solo es necesario conectar el pin VCC de alimentación a 3 V - 5 V, el pin GND a tierra y el pin de datos a un pin digital de Arduino \citep{WEBSITE:8}. 

\begin{figure}[htbp]
	\centering
	\includegraphics[width=.7\textwidth]{./Figures/dht11.jpg}
	\caption{Modelo y dimensiones del sensor DHT11. }

	\label{fig:dht11}
\end{figure}

Las características más importantes del sensor son:

\begin{itemize}
\item Tensión de operación: 3 V - 5 V DC.
\item Rango de medición de temperatura: 0 a 50 °C.
\item Precisión de medición de temperatura: ±2.0 °C.
\item Resolución temperatura: 0,1 °C.
\item Rango de medición de humedad: 20\% a 90\% RH.
\item Precisión de medición de humedad: 5\% RH.
\item Resolución humedad: 1\% RH
\item Tiempo de sensado: 1 seg.
\item Interface digital: \emph{Single-bus}.
\item Modelo: DHT11
\item Peso: 1 g.

\end{itemize}

\subsection{Sensor de corriente AC SCT-013-030}

La familia de sensores SCT-013 está compuesta por sensores de corriente no invasivo que permiten medir la intensidad que atraviesa un conductor sin necesidad de cortar o modificar el conductor. Estos sensores son usados para medir la intensidad o potencia consumida por una carga eléctrica. Los sensores SCT-013 son transformadores de corriente, dispositivos de instrumentación que hacen posible una medición proporcional a la intensidad que atraviesa un circuito. La medición se realiza por inducción electromagnética \citep{WEBSITE:9}. 

Los sensores SCT-013 disponen de un núcleo ferromagnético partido (como una pinza) que permite abrirlo para arrollar un conductor de una instalación eléctrica sin necesidad de cortarlo, como se ilustra con la figura \ref{fig:sensorCorriente}. Dentro de la familia SCT-013 existen modelos que proporcionan la medición como una salida de intensidad o de tensión. 


\begin{figure}[htbp]
	\centering
	\includegraphics[width=1.0\textwidth]{./Figures/sensorCorriente2.png}
	\caption{Dimensiones del sensor SCT-013-030 AC \protect\footnotemark.}
	\label{fig:sensorCorriente}
\end{figure}

\footnotetext{Imagen de \url{https://datasheetspdf.com/pdf-file/1004704/XiDiTechnology/SCT-013-030/1}}

%El sensor SCT-013 es muy fácil de manejar y acoplar. Puede colocarse como una pinza alrededor de un cable que entre al edificio sin la necesidad de realizar algún trabajo de alta tensión, adecuado para la medición de corriente AC, monitoreo y protección de motores AC, equipo de iluminación, etc \citep{WEBSITE:10}.

Los sensores de la serie SCT-013 son sensores que trabajan como transformadores, la corriente que circula por el cable que se desea medir actúa como el devanado primario (1 espira) e internamente tiene un devanado secundario que dependiendo del modelo puede tener hasta más de 2000 espiras. La cantidad de espiras representa la relación entre corriente que circula por el cable y la que el sensor entrega. Esta relación o proporción es lo que marca la diferencia entre los diferentes modelos SCT-013, adicionalmente pueden tener una resistencia de carga en la salida y de esta forma en lugar de corriente se trabaja con una salida de tensión \citep{WEBSITE:21}. La figura \ref{fig:espiras} ilustra los devanados mencionados:

\begin{figure}[htpb]
\centering 
\includegraphics[width=1.0\textwidth]{./Figures/espiras.jpg}
\caption{Partes del núcleo ferromagnético del sensor de corriente.}
\label{fig:espiras}
\end{figure}

Las especificaciones más importantes por las cuales se escogió este instrumento para el desarrollo del trabajo son:

\begin{itemize}
\item Corriente de entrada (inducción): 0-30 A AC.
\item Modo de salida: 0 - 1 V.
\item No linealidad: ±1%.
\item Resistencia (RL): 62 $\Omega $.
\item Grado de Resistencia: Grade B.
\item Temperatura de operación: -25 °C - ﹢70 °C.
\item Longitud del cable: 1 m.
\item Tamaño abierto: 13 mm x 13 mm.
\end{itemize}

%La precisión del sensor puede ser del 1\% a 2\%, pero para ello es muy importante que el núcleo ferromagnético se cierre adecuadamente. Hasta un pequeño hueco de aire puede introducir desviaciones del 10\%. Como desventaja, al ser una carga inductiva, el SCT-013 introduce una variación del ángulo de fase cuyo valor es función de la carga que lo atraviesa, pudiendo llegar a ser de hasta 3º \citep{WEBSITE:9}.

%Los sensores SCT-013 son pequeños transformadores de corriente o CT (\emph{Current transformartor}) instrumentos ampliamente empleados como elementos de medición. Un transformador de corriente es similar a un transformador de tensión y está basado en los mismos principios de funcionamiento. Sin embargo, persiguen objetivos diferentes \citep{WEBSITE:9}.



%La relación de transformación de intensidad se expresa entre el número de espiras, como se muestra en ecuación \ref{eq:proporcionform}.

%\begin{equation}
%	\label{eq:proporcionform}
%	\left( \frac{Is}{Ip} \right)=\left( \frac{Vp}{Vs} \right)=\left( \frac{Np}{Ns} \right)
%\end{equation}

%A esto se le llama relación de transformación. Relaciona el número de espiras del devanado primario (Np), del devanado secundario (Ns), las intensidades del primario (Ip), del secundario (Is), la tensión del primario (Vp) y del secundario (Vs). 


%Para utilizar el sensor SCT-013 no es necesario interrumpir (cortar o desempalmar) el cable a medir, porque al igual que una pinza amperimétrica tiene el núcleo partido. Si se pasa los dos cables de una conexión monofásica, la lectura será 0, puesto que los cables tienen corrientes opuestas \citep{WEBSITE:21}. La figura \ref{fig:conectacorrecto} ilustra la forma correcta de uso del sensor.

%\vspace{0.5cm}
%\begin{figure}[htpb]
%\centering 
%\includegraphics[width=0.8\textwidth]{./Figures/correcto.jpg}
%\caption{Forma de uso del sensor de corriente \protect\footnotemark.}
%\label{fig:conectacorrecto}
%\end{figure}

%\footnotetext{Imagen de \url{https://programarfacil.com/blog/arduino-blog/sct-013-consumo-electrico-arduino/}}


A diferencia de los transformadores de tensión, en un transformador de intensidad el circuito secundario nunca debe estar abierto, porque las corrientes inducidas pueden llegar a dañar el componente. Por este motivo, los sensores SCT-013 disponen de protecciones (resistencia de Burden en los sensores de salida por tensión, o diodos de protección en los sensores de salida por corriente)\citep{WEBSITE:9}.

Para el trabajo se consideró el módulo SCT-013 de 30 A y con soporte para 250 VAC.
%%%%%%%%%%%%%%%%%%%%%%%%%%%%%%%%%%%%%%%%%%%%
\subsection{Sensor de tensión eléctrica AC - ZMPT101B}

El sensor transformador de tensión ZMPT101B permite medir tensión alterna como la que se dispone en la red hogareña. Esta tensión AC no puede ser medida directamente por el ADC del circuito o placa debido a que escapa del rango de entrada (0 V a 5 V). El módulo ZMPT101B soluciona el problema reduciendo la tensión AC de entrada a una tensión menor que pueda ser leída por el Arduino, NodeMCU o cualquier otro microcontrolador. La figura \ref{fig:sensortension} ilustra la forma de conexión del módulo sensor a la línea eléctrica doméstica.

El sensor está integrado por un transformador que cumple la función de aislador galvánico para mayor seguridad en el uso. El lado primario del transformador se conecta a la tensión alterna que se desea medir, por ejemplo la red eléctrica de un hogar de 220 VAC. En el lado secundario del transformador se encuentra un divisor de tensión y un circuito con amplificador operacional (OPAMP LM358) para adicionar un desplazamiento (\emph{offset}) a la salida análoga \citep{WEBSITE:22}. 

\begin{figure}[htpb]
\centering 
\includegraphics[width=0.95\textwidth]{./Figures/sensortension.png}
\caption{Conexión y partes del sensor de tensión.}
\label{fig:sensortension}
\end{figure}


Este sensor soporta una tensión de entrada de hasta 250 VAC y entrega una onda senoidal de amplitud regulable por un potenciómetro en placa. La onda senoidal de salida está desplazada positivamente para que no tenga tensiones negativas y así poder leerla completamente con el ADC. El desplazamiento depende de la tensión con la que se alimenta el módulo sensor: si la tensión de alimentación es de 5 V el desplazamiento será de 2,5 V y si se alimenta el módulo con 3,3 V el desplazamiento será de 1,65 V \citep{WEBSITE:23}. Su circuito de acondicionamiento de señal interno permite que la tensión de salida del módulo sensor pueda ser leído por cualquier microcontrolador con entrada analógica (ADC). De esta forma, es posible leer la tensión instantánea y realizar cálculos de energía como tensión pico a pico (Vpp) y tensión eficaz (Vrms). 

%Debido a la naturaleza de los transformadores, solo puede medir tensión AC \citep{WEBSITE:22} \citep{ARTICLE:1}. Para su uso con la tarjeta NodeMCU ESP8266, los extremos son 0 y 3.3 V con una compensación de 1.65 V. La figura \ref{fig:ondas} muestra el desplazamiento de onda senoidal.


%\begin{figure}[htpb]
%\centering 
%\includegraphics[width=1.02\textwidth]{./Figures/ondas.png}
%\caption{Señal del ZMPT101B con el NodeMCU ESP8266 \protect\footnotemark.}
%\label{fig:ondas}
%\end{figure}

%\footnotetext{Imagen tomada de \url{https://surtrtech.com/2020/04/08/}}

%%%%%%%%%%%%%%%%%%%%%%%%%%%%%%%%%%%%%%%%%%%%



\subsection{Relé Actuador}

Un relé es un interruptor que se puede activar mediante una señal eléctrica. En su versión más simple es un pequeño electro-imán que cuando se lo excita mueve la posición de un contacto eléctrico de conectado a desconectado o viceversa para accionar un circuito mayor. 

Los relés más utilizados son módulos capaces de activarse mediante la entrada de 5 V. La elección de un módulo relé adecuado dependerá de la tensión y amperaje que debe gestionar. En la figura \ref{fig:rele} se aprecian dos modelos de módulos relé con capacidad de activación 5 V y con soporte para 10 A - 250 VAC y 30 A – 250 VAC respectivamente.


\begin{figure}[htbp]
	\centering
	\includegraphics[width=1.0\textwidth]{./Figures/rele.jpg}
	\caption{Modelos de relés con activación de 5 V.}

	\label{fig:rele}
\end{figure}

%El relé es un interruptor que permite trabajar con dos circuitos, uno con tensiones elevadas, por ejemplo, 220 V pero que es activado por un circuito de tensión inferior, por ejemplo, 5 V.

Para el trabajo se consideró el módulo relé de activación 5 V y con soporte para 30 A - 250 VAC.

\subsection{Lenguajes de programación}

La elaboración de este trabajo involucró el usó de distintos \emph{software}s como herramientas para facilitar el desarrollo, así como el uso de diversos lenguajes de programación que se describen a continuación:
\begin{itemize}
\item Python: es un lenguaje de programación interpretado cuya filosofía hace hincapié en la legibilidad de su código. Se trata de un lenguaje de programación multiparadigma, ya que soporta parcialmente la orientación a objetos, programación imperativa y, en menor medida, programación funcional.

Se utilizó para la creación de procesos internos en el módulo principal. 
\item PHP: es un lenguaje de programación de uso general que se adapta especialmente al desarrollo web del lado del servidor.

Se utilizó como lenguaje \emph{backend} del \emph{software} de monitoreo y control.
\item JavaScript: lenguaje de programación interpretado utilizado en el lado del cliente. Es el único lenguaje de programación que funciona en los navegadores de forma nativa.

Se utilizó como lenguaje frontend del \emph{software} de monitoreo y control.
\item Arduino: lenguaje de programación que está basado en C++.

Se utilizó como lenguaje para programar el firmware de los módulos de sensores y actuadores.
\end{itemize} 
\chapter{Diseño e implementación} % Main chapter title

\label{Chapter3} % Change X to a consecutive number; for referencing this chapter elsewhere, use \ref{ChapterX}

En este capítulo se detallan las consideraciones de diseño que se tuvieron en cuenta para realizar el sistema IoT y se describe el desarrollo y servicios implementados para este trabajo.



%----------------------------------------------------------------------------------------
%	SECTION 1
%----------------------------------------------------------------------------------------


\section{Diseño general del sistema}

Como se observa en la figura \ref{fig:arquitectura}, el sistema está compuesto por una arquitectura que integra cuatro componentes necesarios para su funcionamiento, los dispositivos (sensores y actuadores), el servidor local, los elementos de red y el servicio en la nube.

El diseño del sistema se basó en una arquitectura distribuida donde se despliegan distintas tecnologías de \emph{hardware} y \emph{software} con el objetivo de ofrecer acceso desde dentro o fuera de la red doméstica del edificio. La interfaz de comunicación con el servidor interno y externo se hizo mediante el protocolo MQTT por ser ligero y de bajo consumo de ancho de banda para la red \emph{wireless} utilizada. 

La etapa de diseño se inició con la modelización de los componentes más importantes del sistema IoT, que son los módulos sensores y actuadores. En un segundo paso se desarrolló el software de monitoreo web, según patrones de diseño y características para garantizar la seguridad de la aplicación web. 

Para el diseño e implementación de \emph{firmware} se utilizó Arduino teniendo en cuenta patrones y conceptos de la programación funcional. La comunicación MQTT se realizó mediante la biblioteca PubSubClient versión 2.7.0 \citep{WEBSITE:46} y la visualización gráfica de las pantallas OLED y GLCD, con las bibliotecas Adafruit GFX versión 1.7.2 \citep{WEBSITE:47} y U8g2\_for\_Adafruit\_GFX versión 2.28.10 \citep{WEBSITE:48} respectivamente.


Como se eligió el protocolo MQTT para la comunicación entre módulos y el conjunto de servicios, se definieron tres tipos de tópicos para poder comunicarse entre sí. Estos tópicos son:

\begin{itemize}

\item Tópico de envió de valores: permite que un módulo envíe un dato JSON con valores de su sensor hacia el servidor local.  La figura \ref{fig:comunica1} muestra la lógica de su aplicación en la comunicación.
%Solamente utilizado mediante comunicación unidireccional.

\item Tópico de comunicación de estados: es utilizado por todos los módulos y permite el envió de un dato JSON con valores de su estado actual desde el sensor o actuador hacia el servidor local. La figura \ref{fig:comunica1} muestra la lógica de su aplicación en la comunicación.
%Solamente utilizado mediante comunicación unidireccional.

%%%%%%%%%%%%%%%%%%%%%%%%%%% imagen horizontal%%%%%%%%%%%%%%%%%%%%%%%%%%%%%%%%%%%%%%%%%%%%
\begin{landscape} % esto es para rotar la pagina e imagen
\begin{figure}[htpb]
\centering 
\includegraphics[width=1.65\textwidth]{./Figures/arquitectura-listo.png}
\caption{Arquitectura del sistema IoT.}
\label{fig:arquitectura}
\end{figure}
\end{landscape} % esto es para rotar
%%%%%%%%%%%%%%%%%%%%%%%%%%%%%%%%%%%%%%%%%%%%%%%%%%%%%%%%%%%%%%%%%%%%%%%%%%

%\vspace{1cm}
\begin{figure}[htpb]
\centering 
\includegraphics[width=0.6\textwidth]{./Figures/sensores.png}
\caption{Lógica de comunicación MQTT del módulo sensor.}
\label{fig:comunica1}
\end{figure}


\item Tópico de sincronización: permite que un módulo envíe un dato JSON para actualizar información en todos los módulos suscriptos al tópico. Es utilizado en la comunicación entre el \emph{software} de monitoreo, los actuadores y el broker del servidor local. La figura \ref{fig:comunica2} muestra la lógica de su aplicación en la comunicación.
\end{itemize}


\begin{figure}[htpb]
\centering 
\includegraphics[width=0.6\textwidth]{./Figures/actuadores.png}
\caption{Lógica de comunicación MQTT del módulo actuador.}
\label{fig:comunica2}
\end{figure}




El software de monitoreo de tipo web local y remoto fue diseñado y desarrollado a medida para garantizar dominio total de la plataforma usada. Utiliza el \emph{framework bootstrap} para lograr interfaces gráficas de usuario que cumplan con los principios de experiencia de usuario (UX) y compatibilidad de dispositivos. Para la recepción de mensajes se usó web sockets mediante \emph{JavaScript} teniendo configurado los tres tópicos de comunicación antes mencionados para el intercambio de mensajes con los módulos.

La figura \ref{fig:diagrama_general} muestra el diagrama general del funcionamiento del sistema desde la perspectiva conceptual. El desarrollo, consideraciones de construcción y fabricación para cada módulo del sistema se describen en detalle en las siguientes secciones de este capítulo.


\begin{figure}[htbp]
	\centering
	\includegraphics[width=1.0\textwidth]{./Figures/diagrama0.png}
	\caption{Diagrama general del funcionamiento del sistema.}

	\label{fig:diagrama_general}
\end{figure}

%\vspace{0.5cm}


%\section{Medidas de ciberseguridad}

%Los requerimientos de ciberseguridad dentro del desarrollo ocupan un lugar muy importante en cada una de las etapas ejecutadas en el proceso de implementación de un sistema IoT, porque permite garantizar un grado mínimo de seguridad y confiabilidad funcional del producto. 

%Los requerimientos considerados durante el proceso, son los siguientes:





%\subsection{Requerimientos para los módulos IoT}
%Se buscó cumplir con los requerimientos que se plantean a continuación:

%\begin{itemize}
%\item Uso de programación basada en código modular para el desarrollo del \emph{firmware}.
%\item Uso de la biblioteca en su versión más actual para la comunicación MQTT.
%\item Los objetos de datos a transmitir serán del formato JSON.
%\item La comunicación el protocolo MQTT debe contener TLS.
%\end{itemize}





%\section{Construcción y programación de módulos}

%En esta sección se describen el proceso y consideraciones técnicas para la construcción de cada uno de los módulos del sistema IoT propuesto.
%%%%%%%%%%%%%%%%%
\section{Módulo principal}

El módulo principal representa el elemento central dentro de la solución IoT planteada y para  su construcción e instalación se utilizaron recursos que hicieron posible una versión de fácil uso para el usuario. Las consideraciones generales de seguridad para el módulo principal fueron:

\begin{itemize}
\item Sistema operativo GNU/Linux oficial Raspberry Pi OS \citep{WEBSITE:44}\citep{WEBSITE:45}.
\item Acceso al sistema operativo mediante usuario y contraseña.
\item Accesos remotos por SSH (\emph{Secure Shell}) y FTP (\emph{File Transfer Protocol})  desactivados.
\item Cifrado de las unidades de almacenamiento del sistema operativo.
\end{itemize}

%La figura \ref{fig:argon} muestra la integración y ensamblado del módulo.

%%%%%%%%%%%%%%%%%%%%%%%%%%% imagen horizontal%%%%%%%%%%%%%%%%%%%%%%%%%%%%%%%%%%%%%%%%%%%%
%\begin{landscape} % esto es para rotar la pagina e imagen
%\begin{figure}[htpb]
%\centering 
%\includegraphics[width=1.7\textwidth]{./Figures/argon.png}
%\includegraphics[width=0.92\textwidth]{./Figures/armadoactuador.png}
%\caption{Ensamblado y partes del módulo principal. }
%\label{fig:argon}
%\end{figure}
%\end{landscape} % esto es para rotar
%%%%%%%%%%%%%%%%%%%%%%%%%%%%%%%%%%%%%%%%%%%%%%%%%%%%%%%%%%%%%%%%%%%%%%%%%%%





%Las consideraciones de seguridad para la configuración del broker fueron:

%\begin{itemize}
%\item Configuración de usuario y contraseña para controlar el acceso a los canales de comunicación del broker local y remoto.
%\item Uso de canales separados, para el envió, sincronización y respuesta entre elementos del sistema IoT.
%\item Cada mensaje debe ir con destino a un tópico en específico, evitar envió de datos a la instancia general \#.
%\item Configurar permisos para la edición o ejecución a los archivos de configuración del broker.
%\end{itemize}


\section{Módulo replicador a la nube}

El módulo replicador a la nube está dentro del módulo principal y para su desarrollo se diseñó una estructura interna compuesta por subprocesos que, al trabajar en conjunto, forman el sistema completo de replicación.

La replicación solo se da mientras exista conexión a Internet. El desarrollo de cada subproceso se programó en Python por tratarse de un lenguaje multiplataforma, robusto y orientado a objetos. La figura \ref{fig:logicareplicador} ilustra la lógica de trabajo del replicador.


%\begin{figure}[htpb]
%\centering 
%\includegraphics[width=1.15\textwidth]{./Figures/replicador.png}
%\caption{Flujo funcional del módulo replicador.}
%\label{fig:flujoreplicador}
%\end{figure}



Las descripciones de cada subproceso interno que contiene el módulo replicador se detallan a continuación usando el nombre del proceso con el cual fue creado: 

\begin{itemize}
\item \keyword{mqtt\_envia\_nube\_poo (P1)}: es el responsable de reenviar al broker remoto todos los mensajes que llegan al broker local. Se usa el formato JSON.

\item \keyword{mqtt\_recibe\_nube\_poo (P2)}: es el responsable de recibir los datos JSON que fueron generados en la aplicación web remota y que llegan desde el broker remoto para luego enviarlos al broker local.

\item \keyword{actuador\_registros\_bd (P3)}: es el responsable de verificar los registros en la base de datos. La verificación se realiza en intervalos de tiempo de una hora. Consulta todos los registros de lecturas de actuadores dentro de una hora específica en las tablas auxiliares de actuadores y consumos, luego obtiene la media de los consumos y hace un solo registro en la tabla de historial de consumo. Finalmente realiza el borrado de los registros temporales utilizados.

\item \keyword{sensor\_registros\_bd (P4)}: es el responsable de verificar los registros en la base de datos. La verificación se realiza en intervalos de tiempo de una hora, consulta todos los registros de lecturas de sensores dentro de una hora específica en las tablas auxiliares de sensores, luego obtiene la media de los registros y hace un solo registro en la tabla de historial de sensores. Finalmente realiza el borrado de los registros temporales utilizados.

\item \keyword{sensor\_historial\_replicas\_bd (P5)}: es el responsable de verificar de forma constante la conexión a Internet y si existen datos por replicar. En caso de disponer de una conexión a Internet y a partir de registros marcados como no replicado procede a enviar los últimos registros hacia la nube. Finalmente actualiza en el campo de la tabla de registros\_no\_enviados, y así se mantiene la consistencia necesaria entre el sistema local y remoto.

\item \keyword{mqtt\_gestionBD\_poo (P6)}: es el responsable de recibir todos los mensajes que llegan al broker y verificar la pertenencia del JSON capturado (sensor o actuador). Posteriormente, comprueba si existe conexión a Internet. En el caso que no esté disponible la conexión solo se hace el registro en la tabla local y se lo marca como no replicado, para que pueda ser enviado a la nube cuando vuelva a existir la conexión a Internet. Este subproceso también permite actualizar el estado de un sensor o actuador en la base de datos con el estado de ``CONECTADO'' mientras esté activo en la red.

%\vspace{0.05cm}
%%%%%%%%%%%%%%%%%%%%%%%%%%% imagen horizontal%%%%%%%%%%%%%%%%%%%%%%%%%%%%%%%%%%%%%%%%%%%%
\begin{landscape} % esto es para rotar la pagina e imagen
\begin{figure}[htbp]
	\centering
	\includegraphics[width=1.2\textwidth]{./Figures/diagrama2.png}
	\caption{Diagrama funcional del replicador. }

	\label{fig:logicareplicador}
\end{figure}
\end{landscape} % esto es para rotar
%%%%%%%%%%%%%%%%%%%%%%%%%%%%%%%%%%%%%%%%%%%%%%%%%%%%%%%%%%%%%%%%%%%%%%%%%%%

\item \keyword{mqtt\_gestionDispositivosConectados (P7)}: es el responsable de recibir todos los mensajes que llegan al broker, verificar la pertenencia del JSON capturado (sensor o actuador) para registrar temporalmente el tiempo de llegada del mensaje del dispositivo. 

Este subproceso usa hilos en Python para estar constantemente registrando los tiempos de llegada de mensajes de cada sensor o actuador, y si los intervalos de tiempo de llegada de mensajes para un dispositivo activo son mayores a un minuto, el subpoceso actualiza el estado del dispositivo con el estado de ``DESCONECTADO'' en la base de datos.

\item \keyword{sensorEstadoRed (P8)}: es el responsable de verificar el estado del servicio de Internet en la red WLAN. Este subproceso usa hilos en Python para registrar periódicamente en la base de datos el estado actual de la red interna.

\end{itemize}

\section{Módulo de medición de temperatura}

Este módulo permite recoger lecturas de temperatura y humedad en ambientes de un hogar, oficina o edificio. Los valores son enviados y procesados en el sistema IoT de control y monitoreo que se encuentra en el servidor web del módulo principal. Las lecturas de temperatura son utilizadas para conocer la curva de cambios de temperatura según el horario registrado, así como su relación directa con el consumo eléctrico por el uso de ventiladores y equipos de aire acondicionado. Para su construcción se usó la placa NodeMCU8266 V3, por su capacidad de conexión inalámbrica. 

Este módulo integra una pantalla SSD1306 OLED para visualizar el valor de la temperatura en tiempo real. Para el encapsulado y construcción se utilizó un tablero adosable de montaje de interruptores térmicos y diferenciales tipo RIEL-DIN, de material de poliestireno y cubierta trasparente de policarbonato con apertura vertical \citep{WEBSITE:17}, como se aprecia en la figura \ref{fig:casetemp}.


\begin{figure}[htpb]
\centering 
\includegraphics[width=0.7\textwidth]{./Figures/casetemp.png}
\caption{Case del módulo de temperatura \protect\footnotemark.}
\label{fig:casetemp}
\end{figure}

\footnotetext{Imagen tomada de \url{https://www.promart.pe/tablero-2-polos-adosable-c-puerta/p}}

El diseño de integración de componentes electrónicos se realizó en una Placa PCB perforada siguiendo el esquemático de la figura \ref{fig:citemp}.

\begin{figure}[htpb]
\centering 
\includegraphics[width=0.85\textwidth]{./Figures/ci-temp.png}
\caption{Esquemático electrónico del módulo de temperatura. }
\label{fig:citemp}
\end{figure}

\vspace{0.5cm}
El proceso de integración se puede ver en la figura \ref{fig:entemp}.

\begin{figure}[htpb]
\centering 
\includegraphics[width=0.95\textwidth]{./Figures/temperatura.jpg}
\caption{Ensamblado del módulo de temperatura. }
\label{fig:entemp}
\end{figure}

\section{Módulo actuador}

Este módulo permite activar o desactivar el paso de la corriente eléctrica dentro de un tomacorriente. La acción de cambio de estado (activado o desactivado) se realiza desde un switch en la interfaz de la aplicación web de monitoreo y control. El software envía un mensaje JSON por medio del canal de sincronización con el código y número de modelo del relé a activar o desactivar.

Para la construcción del módulo se utilizó una caja de tomacorriente  resistente a impactos, de gran durabilidad, autoextinguible e ideal para conductos de cables \citep{WEBSITE:18}. Para fijar el tomacorriente se usó una placa modular de soporte. En la figura \ref{fig:caseactuador} se ilustran los componentes mencionados.

%El corte o paso de energía eléctrica se realiza desde el software de monitoreo mediante el componente \emph{Toggle Switch} para activar o desactivar el relé. 

Para la activación del relé de 5 V mediante la salida de la placa NodeMCU8266 fue necesario usar un convertidor de tensión DC-DC Step-Up 2 A MT3608, porque las salidas de la placa NodeMCU8266 son de 3,3 V y la activación del relé requiere 5 V para funcionar. La figura \ref{fig:esquemaactuador} ilustra el relé de 30 A y el convertidor de tensión utilizado para la solución a este problema.


%El DC-DC Step-Up tiene como función entregar una tensión de salida constante superior a la tensión de entrada, soporta como tensión de entrada entre 2 V a 24 V y tensión de salida entre 2 V a 28 V. La tensión de salida se puede regular mediante un potenciómetro multivuelta \citep{WEBSITE:19}. 


\begin{figure}[htpb]
\centering 
\includegraphics[width=1.0\textwidth]{./Figures/actuador.jpg}
\caption{Case del módulo actuador.}
\label{fig:caseactuador}
\end{figure}


\begin{figure}[htpb]
\centering 
\includegraphics[width=1.0\textwidth]{./Figures/esquemaactuador.png}
\caption{Uso del Step-Up DC-DC para activación del relé. }
\label{fig:esquemaactuador}
\end{figure}

%\vspace{1cm}
%\vspace{1cm}

\section{Módulo de consumo eléctrico}

Este módulo es el responsable de medir el consumo de energía eléctrica dentro de un hogar, oficina o edificio. Este módulo fue desarrollado para permitir la comunicación bidireccional entre el módulo y el servidor local por la necesidad de sincronización con todos los clientes actuadores conectados al sistema.

%\vspace{2cm}

\keyword{Cálculo de consumo de energía eléctrica}

La energía que consume un artefacto eléctrico, se determina multiplicando la potencia de dicho artefacto por la cantidad de horas que está encendido \citep{BOOK:3}. Por ejemplo ver la ecuación \ref{eq:consumoform}.

\begin{equation}
	\label{eq:consumoform}
	EC = \left( P \cdot T \right)
\end{equation}

\vspace{0.1cm}
Siendo las variables y unidades:
\begin{itemize}
\item EC: energía consumida (kWh)
\item T: tiempo que esta encendido (h)
\item P: potencia eléctrica del artefacto (kW)
\end{itemize}

\vspace{0.1cm}
\keyword{Cálculo de la potencia eléctrica}

El cálculo de la potencia eléctrica se obtiene multiplicando la carga eléctrica, también conocida como tensión eléctrica, que pasa en un instante de tiempo a través de una diferencia de potencia, denominada intensidad. El resultado, cuya unidad es el vatio (en inglés, watt) su símbolo es la W, se obtiene al multiplicar la tensión por la intensidad. La tensión se pone en Voltios (V) y la intensidad en Amperios (A). La fórmula de la potencia eléctrica se ilustra en la ecuación \ref{eq:potenciaform} \citep{WEBSITE:20}.

\begin{equation}
	\label{eq:potenciaform}
	P = \left( V \cdot I \right)
\end{equation}

\vspace{0.2cm}
Siendo las variables y unidades:
\begin{itemize}
\item V: tension eléctrica (V)
\item I: intensidad eléctrica (A)
\item P: potencia eléctrica del artefacto (W)
\end{itemize}


Como se observa en la ecuación \ref{eq:potenciaform}, para poder medir el consumo eléctrico se necesita medir la tensión (V) y la intensidad (A), para lo que se utilizaron los sensores SCT-013-030 y  AC - ZMPT101B, respectivamente.

Los aspectos más importantes para el diseño, desarrollo y construcción del módulo de consumo eléctrico se describen  a continuación:


\begin{enumerate}
\item \keyword{Componentes para la construcción del módulo}

Los elementos necesarios fueron:
\begin{itemize}
\item Gabinete de protección.
\item Sensor de tensión eléctrica AC - ZMPT101B.
\item Sensor de corriente eléctrica SCT-013-030.
\item Convertidor ADC ADS1115.
\item Cable de calibre 14.
\item Fusible de 30 A.
\item Pantalla gráfica LCD.
\item Fuente embebida con entrada 220 V y salida de 5 V.
\item Leds ultra brillantes.
\end{itemize}

\item \keyword{Aplicación del sensor SCT-013-030}

Para este trabajo se utilizó el sensor de salida por tensión SCT-013-030 que soporta corrientes máximas de 30 A (30 A /1 V) y salida en tensión de 1 V.A. Una intensidad de 30 A a 230 V corresponde con una carga de 6.900 W, potencia suficiente para la mayoría de usuarios domésticos. En la figura \ref{fig:consumo1} se ilustra el esquema lógico a usar y la pinza del sensor.
\vspace{0.5cm}
\begin{figure}[htpb]
\centering 
\includegraphics[width=0.9\textwidth]{./Figures/consumo1.png}
\caption{Circuito y pinza del sensor de corriente. }
\label{fig:consumo1}
\end{figure}

El proceso de ensamblado se ilustra en las fotografías de la figura \ref{fig:armadoactuador}.

\begin{figure}[htpb]
\centering 
\includegraphics[width=0.85\textwidth]{./Figures/armadoactuador.jpg}
\caption{Ensamblado del módulo actuador. }
\label{fig:armadoactuador}
\end{figure}


\item \keyword{Medición del consumo eléctrico}

Para determinar la potencia eléctrica consumida, el sistema recolecta mediciones del sensor de consumo y realiza la operación matemática de la formula \ref{eq:potenciaform}. Los resultados los almacena de forma periódica (aproximadamente cada 2 segundos) en la base de datos local y remota. Los datos son agrupados según la hora de lectura junto a su respectiva fecha. La tabla \ref{tab:tablaconsumos} muestra la lógica de registros.




\begin{table}[h]
	\centering
	\caption[Registros de consumos]{Registros de consumos}
	\begin{tabular}{l c c }     
		\toprule
		\textbf{Potencia electrodoméstico} & \textbf{Respaldo} &\textbf{Fecha} \\
		\midrule
		Potencia media del ventilador (Pmv) & 10:00 am & 01/02/2022\\		
		Potencia media del ventilador (Pmv) & 11:00 am &01/02/2022 \\
		Potencia media del ventilador (Pmv) & 2:00 pm & 01/02/2022\\		
		Potencia media del ventilador (Pmv) & 3:00 pm & 01/02/2022\\		
		
		\bottomrule
		\hline
	\end{tabular}
	\label{tab:tablaconsumos}
\end{table}

Cada agrupación tiene un aproximado de 1800 registros temporales obtenidos en una determinada hora. Si se considera como ejemplo los datos de la tabla \ref{tab:tablaconsumos}, el consumo del ventilador en el día 01/02/2022 se calcula usando la formula \ref{eq:consumoform}, dando como resultado la ecuación \ref{eq:potenciaformejemplo2}.

\begin{equation}
	\label{eq:potenciaformejemplo2}
	EC =  \left( P \cdot T \right) = \left(Pmv \cdot 4 \right)
\end{equation}

Los valores del consumo total serán usados para generar la facturación mensual por consumo eléctrico. La figura \ref{fig:modconsumo} muestra la construcción del módulo de consumo.

\begin{figure}[htpb]
\centering 
\includegraphics[width=1.0\textwidth]{./Figures/moduloconsumo.png}
\caption{Construcción del módulo de consumo.}
\label{fig:modconsumo}
\end{figure}


\end{enumerate}
\section{Módulo réplica}
Este módulo es una copia del software de monitoreo y control local y esta ubicado en la nube. Utiliza los servicios de un servidor y un broker remoto.

%\vspace{1.0cm}



%%%%%%%%%%%%%%%%%%%%%%%%%%%%%%%%%%%%%%%%%%%%%%%%%%%%%%%%%%%%%%%%%%%%%%%%%%%%%
%\begin{landscape} % esto es para rotar la pagina e imagen
%\begin{figure}[htpb]
%\centering 
%\includegraphics[width=1.5\textwidth]{./Figures/moduloconsumo.png}
%\caption{Construcción del módulo de consumo.}
%\label{fig:modconsumo}
%\end{figure}
%\end{landscape} % 
%%%%%%%%%%%%%%%%%%%%%%%%%%%%%%%%%%%%%%%%%%%%%%%%%%%%%%%%%%%%%%%%%%%%%%%%%%%%%%
\section{Diseño de la red IoT}

Para el diseño de la red se agregó un router inalámbrico como punto de acceso adicional a la red local, sirviendo como medio de comunicación exclusivo de los sensores y actuadores del sistema dentro de la red interna del hogar o edificio. La figura \ref{fig:diagramared} ilustra el diseño físico de la red utilizada.
%\vspace{0.5cm}
%%%%%%%%%%%%%%%%%%%%%%%%%%%%%%%%%%%%%%%%%%%%%%%%%%%%%%%%%%%%%%%%%%%%%%%%%%%%%
%\begin{landscape} % esto es para rotar la pagina e imagen
\begin{figure}[htpb]
\centering 
\includegraphics[width=1.0\textwidth]{./Figures/rediot3.png}
\caption{Diseño físico de la red IoT para el sistema.}
\label{fig:diagramared}
\end{figure}
%\end{landscape} % 
%%%%%%%%%%%%%%%%%%%%%%%%%%%%%%%%%%%%%%%%%%%%%%%%%%%%%%%%%%%%%%%%%%%%%%%%%%%%%
% se retira la seccion de configuración wlan
%%%%%%%%%%%%%%%%%%%%%%%%%%%%%%%%%%%%%%%%%%%%%%%%%%%%%%%%%%%%%%%%%%%%%%%%%%%%%
\section{Diseño del software para monitoreo y control}

El software desarrollado para este trabajo fue diseñado e implementado a medida por ser parte fundamental dentro de los objetivos planteados al inicio. El software es de tipo web y cumple con la característica de ser un diseño responsivo para garantizar la correcta visualización en los distintos dispositivos del mercado actual. La figura \ref{fig:patrondiseniosoftware} muestra el patrón a seguir en el desarrollo de las interfaces gráficas de usuario (GUI) del software.

\begin{figure}[htpb]
\centering 
\includegraphics[width=0.75\textwidth]{./Figures/responsive3.png}
\caption{Patrón del diseño web responsivo para el \emph{software} \protect\footnotemark.}
\label{fig:patrondiseniosoftware}
\end{figure}

\footnotetext{Imagen tomada de \url{https://www.genbeta.com/desarrollo/responsive-design-introduccion}}


\section{Almacenamiento de los datos}

El sistema de monitoreo utiliza una base de datos para almacenar los valores de sensores y consumos. Por consiguiente, para este trabajo fue necesaria la creación de una base de datos para almacenar los valores que se generan por todos los módulos de sensores y actuadores. 

El gestor de base de datos elegido fue MySQL por ser de característica \emph{software} libre y ofrecer alta compatibilidad de conexión con el lenguaje \emph{backend} utilizado. El modelo entidad relación implementado en la base de datos se muestra en la figura \ref{fig:entidadrelacion}:

\begin{figure}[htpb]
\centering 
\includegraphics[width=1.0\textwidth]{./Figures/entidad-relacion.png}
\caption{Modelo entidad relación de la base de datos.}
\label{fig:entidadrelacion}
\end{figure}
% Chapter Template

\chapter{Ensayos y resultados} % Main chapter title

\label{Chapter4} % Change X to a consecutive number; for referencing this chapter elsewhere, use \ref{ChapterX}


%----------------------------------------------------------------------------------------
%	SECTION 1
%----------------------------------------------------------------------------------------
En este capítulo se detallan los resultados esperados y obtenidos sobre cada una de las pruebas realizadas para validar la integración del sistema y poder comprobar que el alcance funcional logrado es acorde a lo esperado.

%\citep{ARTICLE:1}, \citep{BOOK:1}, \citep{BOOK:2}, \citep{WEBSITE:1}.

\section{Banco de pruebas}

Todos los ensayos que se describen en este capítulo fueron efectuados utilizando el diseño físico de red que se muestra en la figura \ref{fig:banco}. Las pruebas funcionales desde dentro de la red local se realizaron con una laptop MacBook Pro y un equipo de escritorio con Windows 10. Las pruebas de funcionalidad remota se realizaron utilizando un dispositivo móvil Samsung A50 con acceso a la red celular mediante el uso de paquetes de datos a Internet.

\begin{figure}[htbp]
	\centering
	\includegraphics[width=0.82\textwidth]{./Figures/banco2.png}
	\caption{Esquema del banco de pruebas utilizado.}

	\label{fig:banco}
\end{figure}

%%%%%%%%%%%%%%%%%%%%%%%%%%%%%%%%%%%%%%%%%%%%%%%%%%%%%%%%%%%%%%%%%%%%%%


\section{Resultados de los módulos del sistema IoT}
En esta sección se muestran las imágenes reales de los resultados obtenidos de la construcción de cada módulo físico. Estos dispositivos fueron utilizados para las pruebas de validación en un ambiente real IoT. Las figuras \ref{fig:modPrincipal}, \ref{fig:modTemp} y \ref{fig:modConsumo2} muestran los módulos.

 
\begin{figure}[htpb]
\centering 
\includegraphics[width=0.9\textwidth]{./Figures/principal2.png}
\caption{Vista superior y posterior del módulo principal.}
\label{fig:modPrincipal}
\end{figure}



\begin{figure}[htpb]
\centering 
\includegraphics[width=0.85\textwidth]{./Figures/moduloTemp2.png}
\caption{Vista lateral, frente y superior del módulo de temperatura.}
\label{fig:modTemp}
\end{figure}



%%%%%%%%%%%%%%%%%%%%%%%%%%%%%%%%%%%%%%%%%%%%%%%%%%%

\begin{landscape} % esto es para rotar la pagina e imagen
\begin{figure}[htpb]
\centering 
\includegraphics[width=1.8\textwidth]{./Figures/consumo3.png}
\caption{Vista frontal y lateral del módulo actuador y módulo de consumo.}
\label{fig:modConsumo2}
\end{figure}
\end{landscape} %


%%%%%%%%%%%%%%%%%%%%%%%%%%%%%%%%%%%%%%%%%%%%%%%%%%%
\section{Resultados del software de monitoreo y control}

Las figuras \ref{fig:software1} y \ref{fig:software2} muestran los resultados responsivos para cada tipo de dispositivo considerado en el desarrollo.

\vspace{0.5cm}
\begin{figure}[htpb]
\centering 
\includegraphics[width=0.62 \textwidth]{./Figures/responsive1.png}
\caption{Vista del software en equipos desktop y laptop.}
\label{fig:software1}
\end{figure}

\vspace{1.0cm}
\begin{figure}[htpb]
\centering 
\includegraphics[width=0.65\textwidth]{./Figures/responsive2.png}
\caption{Vista del software en tableta y celular.}
\label{fig:software2}
\end{figure}

\vspace{1.5cm}

El software de monitoreo y control fue diseñado y desarrollado a medida con el objetivo de poseer los derechos de \emph{Copyright}. Por ser de tipo web requiere solo un navegador web y conexión a la red local o a Internet para poder ser utilizado. 

El software ha sido nombrado como Cenergy IoT System y hace referencia al control de energía eléctrica mediante un sistema IoT.  

La interfaz de control de acceso al sistema se muestra en la figura \ref{fig:gui0}. Las credenciales utilizadas son el número de documento nacional de identidad (DNI) y una contraseña de longitud mínima de 8 caracteres.
\vspace{0.5cm}
%%%%%%%%%%%%%%%%%%%%%%%%%%%%%%%%%%%%%%%%%%%%%%%%%%%
%\begin{landscape} % esto es para rotar la pagina e imagen
\begin{figure}[htpb]
\centering 
\includegraphics[width=1.0\textwidth]{./Figures/gui/0.png}
\caption{Interfaz gráfica de acceso al  \emph{software} de monitoreo y control.}
\label{fig:gui0}
\end{figure}
%\end{landscape} %
\vspace{0.5cm}

La interfaz de monitoreo para sensores o actuadores permite mostrar en tiempo real el comportamiento que tienen en el sistema, así como su estado actual. Si un sensor o actuador posee el estado CONECTADO se podrán observar sus detalles desde la opción ``ver detalles''. Esta opción permite acceder a una vista gráfica más detallada del mismo. 

El software actualmente presenta interfaces para cuatro tipos de consultas, lecturas de temperatura, historial de temperatura, consumo sin facturar y consumo facturado. Cada lectura que se registra como consumo en la base de datos se respalda en agrupaciones de intervalos de horas y los registros que aún no completan las horas, podrán ser consultados desde las opciones lectura de temperatura y consumos sin facturar. Los resultados de las consultas podrán ser exportados en formato excel o pdf o, si se desea, ser impresos directamente. %La figura \ref{fig:gui5} muestra la interfaz de consultas.

%%%%%%%%%%%%%%%%%%%%%%%%%%%%%%%%%%%%%%%%%%%%%%%%%%%
%\begin{landscape} % esto es para rotar la pagina e imagen
%\begin{figure}[htpb]
%\centering 
%\includegraphics[width=0.8\textwidth]{./Figures/gui/5.png}
%\caption{Interfaz gráfica de consultas y reportes de la base de datos.}
%\label{fig:gui5}
%\end{figure}
%\end{landscape} %


%%%%%%%%%%%%%%%%%%%%%%%%%%%%%%%%%%%%%%%%%%%%%%%%%%%

%%%%%%%%%%%%%%%%%%%%%%%%%%%%%%%%%%%%%%%%%%%%%%%%%%%
%\begin{landscape} % esto es para rotar la pagina e imagen
%\begin{figure}[htpb]
%\centering 
%\includegraphics[width=1.55\textwidth]{./Figures/gui/1.png}
%\caption{Dashboard inicial del  \emph{software} de monitoreo y control.}
%\label{fig:gui1}
%\end{figure}
%\end{landscape} %

%%%%%%%%%%%%%%%%%%%%%%%%%%%%%%%%%%%%%%%%%%%%%%%%%%%

%%%%%%%%%%%%%%%%%%%%%%%%%%%%%%%%%%%%%%%%%%%%%%%%%%%
%\begin{landscape} % esto es para rotar la pagina e imagen
%\begin{figure}[htpb]
%\centering 
%\includegraphics[width=1.55\textwidth]{./Figures/gui/2.png}
%\caption{Interfaz gráfica de lista de sensores del sistema.}
%\label{fig:gui2}
%\end{figure}
%\end{landscape} %

%%%%%%%%%%%%%%%%%%%%%%%%%%%%%%%%%%%%%%%%%%%%%%%%%%%

%%%%%%%%%%%%%%%%%%%%%%%%%%%%%%%%%%%%%%%%%%%%%%%%%%%
%\begin{landscape} % esto es para rotar la pagina e imagen
%\begin{figure}[htpb]
%\centering 
%\includegraphics[width=1.5\textwidth]{./Figures/gui/2-1.png}
%\caption{Interfaz gráfica donde se muestran todos los detalles de un sensor.}
%\label{fig:gui2-1}
%\end{figure}
%\end{landscape} %

%%%%%%%%%%%%%%%%%%%%%%%%%%%%%%%%%%%%%%%%%%%%%%%%%%%

%%%%%%%%%%%%%%%%%%%%%%%%%%%%%%%%%%%%%%%%%%%%%%%%%%%
%\begin{landscape} % esto es para rotar la pagina e imagen
%\begin{figure}[htpb]
%\centering 
%\includegraphics[width=1.52\textwidth]{./Figures/gui/3.png}
%\caption{Interfaz gráfica de la lista de sensores de consumo junto a su actuador.}
%\label{fig:gui3}
%\end{figure}
%\end{landscape} %

%%%%%%%%%%%%%%%%%%%%%%%%%%%%%%%%%%%%%%%%%%%%%%%%%%%

%%%%%%%%%%%%%%%%%%%%%%%%%%%%%%%%%%%%%%%%%%%%%%%%%%%
%\begin{landscape} % esto es para rotar la pagina e imagen
%\begin{figure}[htpb]
%\centering 
%\includegraphics[width=1.52\textwidth]{./Figures/gui/3-1.png}
%\caption{Interfaz gráfica de usuario donde se muestran todos los detalles de un sensor de consumo.}
%\label{fig:gui3-1}
%\end{figure}
%\end{landscape} %

%%%%%%%%%%%%%%%%%%%%%%%%%%%%%%%%%%%%%%%%%%%%%%%%%%%

%%%%%%%%%%%%%%%%%%%%%%%%%%%%%%%%%%%%%%%%%%%%%%%%%%%
%\begin{landscape} % esto es para rotar la pagina e imagen
%\begin{figure}[htpb]
%\centering 
%\includegraphics[width=1.55\textwidth]{./Figures/gui/4.png}
%\caption{Interfaz gráfica para agregar un nuevo dispositivo al sistema.}
%\label{fig:gui4}
%\end{figure}
%\end{landscape} %

%%%%%%%%%%%%%%%%%%%%%%%%%%%%%%%%%%%%%%%%%%%%%%%%%%%



%%%%%%%%%%%%%%%%%%%%%%%%%%%%%%%%%%%%%%%%%%%%%%%%%%%
%\begin{landscape} % esto es para rotar la pagina e imagen
%\begin{figure}[htpb]
%\centering 
%\includegraphics[width=1.5\textwidth]{./Figures/gui/nucleo.png}
%\caption{Grafo de comunicación y sincronización del núcleo del sistema IoT.}
%\label{fig:grafo}
%\end{figure}
%\end{landscape} %
%%%%%%%%%%%%%%%%%%%%%%%%%%%%%%%%%%%%%%%%%%%%%%%%%%%%%%%%%%%%%%%%%%%%%%
\vspace{1.0cm}
\section{Pruebas de elección de canal y ancho de banda}
Para las pruebas y análisis de las señales inalámbricas se utilizó el software WiFi Explorer Lite. Es una herramienta de descubrimiento de redes inalámbricas que ayuda a identificar conflictos de canales y problemas de configuración que puedan afectar la conectividad o el rendimiento de la red Wi-Fi de un hogar u oficina \citep{WEBSITE:24}. 

%Los fundamentos y consideraciones para la elección del canal y ancho de banda de la señal Wi-Fi que se utilizó, se describieron en el capítulo 3. La elección dependió de las señales circundantes vecinas a la red doméstica donde se instaló el sistema prototipo IoT. 

La figura \ref{fig:test04} muestra el resultado del primer escaneo de las señales y el uso de los canales respectivos así como el solapamiento existente entre ellos. Las señales importantes son:

\keyword{Señal WLAN IoT} 
\begin{itemize}
\item SSID: MATRIX-ICF.
\item Canal: 10 (configuración automática).
\item Ancho de canal: 40 MHz (configuración automática).
\item Potencia señal: 93\%.
\item Seguridad:  WPA2 (PSK).
\item Tasa máxima de transferencia: 300 Mbps.
\end{itemize}


\keyword{Señal WLAN doméstica}
\begin{itemize} 
\item SSID: CLARO-B612-D514.
\item Canal: 11 (configuración automática).
\item Ancho de canal: 20 MHz (configuración automática).
\item Potencia señal: 64\%.
\item Seguridad: WPA2 (PSK).
\item Tasa Máxima de transferencia: 144,4 Mbps.
\end{itemize}

El procedimiento de mejora de la señal IoT consistió en modificar la configuración por defecto del router/AP al cambiar el canal y reducir el ancho de banda. En cada cambio se verificó el comportamiento de las señales en el ambiente y la reducción de solapamiento de los mismos.

%La figura \ref{fig:test02} muestra el ancho de banda que ocupa el canal de comunicación de la señal del router sin configurar. Esta configuración por defecto demuestra no ser la más adecuada para el ambiente IoT debido a que produce interferencias a las señales circundantes.

%La figura \ref{fig:test03} muestra las características de la señal inalámbrica doméstica del lugar donde se implementó el sistema IoT prototipo.

%%%%%%%%%%%%%%%%%%%%%%%%%%%%%%%%%%%%%%%%%%%%%%%%%%%%%%%%%%%%%%
El objetivo de usar un software de exploración Wi-Fi es detectar las zonas y canales con mayor interferencia y elegir un canal que tenga la mínima o ninguna intersección con la zona critica. El \emph{software} detectó la zona con mayor solapamiento y lo marcó en color rojo, como se muestra en la figura \ref{fig:test04}, asociado al SSID y al canal que lo causa.

El análisis de las imágenes de señales que genera el \emph{software} de exploración permite conocer cuales podrían ser los canales ideales. El resultado del cambio de canal para la señal destinada a la comunicación IoT, se muestra en la figura \ref{fig:test05}. 

%Si se compara la figura \ref{fig:test02} (antes) con la figura \ref{fig:test05} (después), se puede observar la diferencia del canal configurado al mostrar la reducción de interferencias con las señales circundantes.

%Al cambiar el canal de la señal IoT, el \emph{software} cambia el color de la zona crítica (de rojo a naranja) en señal que el solapamiento se redujo, demostrando que existe una mejora en la señal de comunicación a utilizar. La figura \ref{fig:test06} muestra el resultado de mejora obtenido.

%Las pruebas y cambios de canal para el router/AP se deben realizar durante la instalación y puesta en marcha del sistema IoT y podrán ser actualizados de acuerdo al cronograma establecido para su mantenimiento. 


%%%%%%%%%%%%%%%%%%%%%%%%%%%%%%%%%%%%%%%%%%%%%%%%%%%
%\begin{landscape} % esto es para rotar la pagina e imagen
%\begin{figure}[htpb]
%\centering 
%\includegraphics[width=1.5\textwidth]{./Figures/wifi/01.png}
%\caption{Estado inicial de las señales Wi-Fi local y circundantes en el ambiente donde se implementó el sistema IoT.}
%\label{fig:test01}
%\end{figure}
%\end{landscape} %

%%%%%%%%%%%%%%%%%%%%%%%%%%%%%%%%%%%%%%%%%%%%%%%%%%%

%\begin{landscape} % esto es para rotar la pagina e imagen
%\begin{figure}[htpb]
%\centering 
%\includegraphics[width=1.5\textwidth]{./Figures/wifi/02.png}
%\caption{Ancho de banda de la señal del router para la red IoT con la configuración por defecto del dispositivo.}
%\label{fig:test02}
%\end{figure}
%\end{landscape} %


%%%%%%%%%%%%%%%%%%%%%%%%%%%%%%%%%%%%%%%%%%%%%%%%%%%
%\begin{landscape} % esto es para rotar la pagina e imagen
%\begin{figure}[htpb]
%\centering 
%\includegraphics[width=1.5\textwidth]{./Figures/wifi/03.png}
%\caption{Ancho de banda  y canal de la señal Wi-Fi doméstica.}
%\label{fig:test03}
%\end{figure}
%\end{landscape} %




%%%%%%%%%%%%%%%%%%%%%%%%%%%%%%%%%%%%%%%%%%%%%%%%%%%

\begin{landscape} % esto es para rotar la pagina e imagen
\begin{figure}[htpb]
\centering 
\includegraphics[width=1.5\textwidth]{./Figures/04.png}
\caption{Zona crítica con mayor interferencia entre los canales de las redes inalámbricas.}
\label{fig:test04}
\end{figure}
\end{landscape} %


%%%%%%%%%%%%%%%%%%%%%%%%%%%%%%%%%%%%%%%%%%%%%%%%%%%

\begin{landscape} % esto es para rotar la pagina e imagen
\begin{figure}[htpb]
\centering 
\includegraphics[width=1.5\textwidth]{./Figures/wifi/05.png}
\caption{Resultado final de ancho de banda y canal sin interferencias críticas.}
\label{fig:test05}
\end{figure}
\end{landscape} %


%%%%%%%%%%%%%%%%%%%%%%%%%%%%%%%%%%%%%%%%%%%%%%%%%%%

%\begin{landscape} % esto es para rotar la pagina e imagen
%\begin{figure}[htpb]
%\centering 
%\includegraphics[width=1.5\textwidth]{./Figures/wifi/06.png}
%\caption{Zona con reducción de solapamiento después de la configuración manual del router.}
%\label{fig:test06}
%\end{figure}
%\end{landscape} %


\section{Pruebas del módulo de temperatura}

El módulo permite leer las variables físicas de temperatura y humedad del ambiente donde se instaló. La temperatura se muestra en la pantalla OLED del módulo y con más detalle en el \emph{software} web de monitoreo y control.

La figura \ref{fig:test-temp} muestra la instalación para las pruebas y el valor obtenido en la pantalla OLED. La figura \ref{fig:temp-lectura} muestra los valores obtenidos en el \emph{software} de monitoreo y sus detalles se muestran en la figura \ref{fig:temp-detalle}.

\begin{figure}[htpb]
\centering 
\includegraphics[width=0.9\textwidth]{./Figures/test/temp/test-temp.png}
\caption{Funcionamiento del módulo de temperatura.}
\label{fig:test-temp}
\end{figure}

La figura \ref{fig:test-panel} muestra las características del panel de visualización del módulo en el software.

\begin{figure}[htpb]
\centering 
\includegraphics[width=0.65\textwidth]{./Figures/test/temp/panel.png}
\caption{Características del panel de visualización.}
\label{fig:test-panel}
\end{figure}

%%%%%%%%%%%%%%%%%%%%%%%%%%%%%%%%%%%%%%%%%%%%%%%%%%%

\begin{landscape} % esto es para rotar la pagina e imagen
\begin{figure}[htpb]
\centering 
\includegraphics[width=1.7\textwidth]{./Figures/test/temp/lectura.png}
\caption{Monitoreo de los módulos la temperatura en el software.}
\label{fig:temp-lectura}
\end{figure}
\end{landscape} %
%%%%%%%%%%%%%%%%%%%%%%%%%%%%%%%%%%%%%%%%%%%%%%%%%%%


\begin{landscape} % esto es para rotar la pagina e imagen
\begin{figure}[htpb]
\centering 
\includegraphics[width=1.7\textwidth]{./Figures/test/temp/detalle.png}
\caption{Detalle del módulo de temperatura conectado y activo.}
\label{fig:temp-detalle}
\end{figure}
\end{landscape} %
%%%%%%%%%%%%%%%%%%%%%%%%%%%%%%%%%%%%%%%%%%%%%%%%%%%

\section{Pruebas del módulo actuador}
Para las pruebas se usó un esquema de conexión como se muestra en la figura \ref{fig:test-esquema}, usando un ventilador como electrodoméstico de consumo. 
\vspace{0.5cm}
\begin{figure}[htpb]
\centering 
\includegraphics[width=0.87\textwidth]{./Figures/test/consumo/esquema.png}
\caption{Esquema de conexión para pruebas del módulo actuador.}
\label{fig:test-esquema}
\end{figure}

Al implementar el módulo y realizar las pruebas respectivas se demostró que el sensor AC-ZMPT101B es muy sensible en la captura del valor de la tensión. El valor obtenido se contrastó con un multímetro digital dando una diferencia aproximada de +2 V / -2 V. La figura \ref{fig:test-tension} muestra la comprobación de la tensión.

\begin{figure}[htpb]
\centering 
\includegraphics[width=1.0\textwidth]{./Figures/test/consumo/tension2.png}
\caption{Comparación de la medida de la tensión entre el módulo y el multímetro.}
\label{fig:test-tension}
\end{figure}

El módulo permite leer las variables físicas de tensión e intensidad así como el estado del relé actuador. Las lecturas se muestran en su pantalla gráfica como se muestra en la figura \ref{fig:test-activa1} y en la figura \ref{fig:test-activa2}.
\vspace{0.5cm}
\begin{figure}[htpb]
\centering 
\includegraphics[width=1.0\textwidth]{./Figures/test/consumo/paso.png}
\caption{Módulo con paso de la corriente eléctrica (led rojo indica riesgo eléctrico en la toma de corriente).}
\label{fig:test-activa1}
\end{figure}

\vspace{0.5cm}
\begin{figure}[htpb]
\centering 
\includegraphics[width=1.0\textwidth]{./Figures/test/consumo/bloqueo.png}
\caption{Módulo con bloqueo de la corriente eléctrica (led azul indica sin riesgo eléctrico en la toma de corriente).}
\label{fig:test-activa2}
\end{figure}

\vspace{0.5cm}
\section{Pruebas de consumo de energía eléctrica}

La prueba de consumo eléctrico se efectuó con la ayuda de un ventilador de hogar con las siguientes características:

\begin{itemize}
\item Marca: ELECTROLUX.
\item Tipo:	circuladores de aire con temporizador.
\item Modelo: BFV10.
\item Número de velocidades: 3.
\item Potencia: 30 W.
\end{itemize}

%La figura \ref{fig:ventilador}  muestra las especificaciones técnicas adheridas en su parte posterior del ventilador. En la actualidad este modelo aún se comercializa pero con un incremento en el número de velocidades y en la potencia de consumo. 

%\begin{figure}[htpb]
%\centering 
%\includegraphics[width=0.7\textwidth]{./Figures/test/consumo/ventilador.png}
%\caption{Etiqueta con información técnica del ventilador.}
%\label{fig:ventilador}
%\end{figure}

Según la guía del organismo supervisor de la inversión en energía (OSINERG - Perú) \citep{BOOK:3}, para calcular el consumo eléctrico el valor de la potencia debe ser convertida a Kilowatts (kW), debido a esto se divide la potencia entre 1000. Respecto al ventilador usado en la prueba el valor ideal sería la ecuación \ref{eq:potenciatest2}:

\begin{equation}
	\label{eq:potenciatest2}
	PE = \left( 0.03 \right) kW
\end{equation}

Para comparar el valor obtenido en la ecuación \ref{eq:potenciatest2}, el módulo de consumo permite obtener un valor real de la potencia del ventilador mediante la fórmula \ref{eq:potenciaform}. Las variables de tensión e intensidad de corriente son multiplicadas para obtener el valor de potencia instantáneo. La figura \ref{fig:registroPotencia} muestra las lecturas almacenadas en la base de datos del sistema IoT. Si se toma como muestra esos valores, se obtiene un promedio de potencia igual a 31,41 W y una desviación estándar aproximada de 0,09444.
%y en la tabla \ref{tab:tablapotencias} se muestra la comparación del valor ideal con el valor real obtenido desde el módulo de consumo.

\begin{figure}[htpb]
\centering 
\includegraphics[width=0.95\textwidth]{./Figures/test/consumo/lecturas.png}
\caption{Lecturas de potencia almacenadas en la base de datos del sistema IoT.}
\label{fig:registroPotencia}
\end{figure}

%\vspace{1.0cm}

%\begin{table}[h]
%	\centering
%	\caption[Comparativa de registros de potencias obtenidas]{Comparativa de registros de potencias obtenidas.}
%	\begin{tabular}{l c c c}    
%		\toprule
%		\textbf{P. ideal (W)} 	 & \textbf{P. ideal (kW)}  & \textbf{P. real - módulo (W)} &\textbf{P. real módulo (kW)} \\
%		\midrule
%		30 & 0.03 & 31.35 & 0.03135\\		
%		30& 0.03 & 31.55  &0.03155 \\
%		30& 0.03 & 31.47 & 0.03147\\		
%		30& 0.03 & 31.36 & 0.03136\\		
%		30& 0.03 & 31.33 & 0.03133\\
%		\bottomrule
%		\hline
%	\end{tabular}
%	\label{tab:tablapotencias}
%\end{table}

Los valores de las variables físicas de tensión e intensidad,  que son necesarias para calcular la  potencia del ventilador, se muestran en el software de monitoreo y control dentro de un conjunto de paneles según la cantidad de módulos registrados en el sistema IoT.

Para facilitar la comprensión de la interfaz gráfica del software, la figura \ref{fig:test-panel5} muestra las características del panel de visualización del módulo en el \emph{software} cuando el actuador permite el paso de la corriente eléctrica. La figura \ref{fig:test-panel4} muestra las características del panel de visualización del módulo en el \emph{software} cuando el actuador bloquea el paso de la corriente eléctrica .

\begin{figure}[htpb]
\centering 
\includegraphics[width=1.0\textwidth]{./Figures/test/consumo/panel5.png}
\caption{Panel de visualización del módulo con paso de corriente eléctrica.}
\label{fig:test-panel5}
\end{figure}

%\vspace{0.5cm}

\begin{figure}[htpb]
\centering 
\includegraphics[width=1.0\textwidth]{./Figures/test/consumo/panel4.png}
\caption{Panel de visualización del módulo con bloqueo de la corriente eléctrica.}
\label{fig:test-panel4}
\end{figure}

Las variables de tensión, intensidad y potencia eléctrica del electrodoméstico conectado, se pueden visualizar en el \emph{software} de monitoreo y control, al igual que los detalles de su funcionamiento en tiempo real. La figura \ref{fig:dashboard-v1} y \ref{fig:dashboard-v2} ilustran lo mencionado.

Para calcular el monto a pagar por el usuario se consideran consumos facturados y consumos no facturados. Los consumos facturados son aquellos registros promedio de un conjunto de registros temporales que se dieron dentro de un intervalo de tiempo. Por ejemplo, dentro del intervalo ``28-03-2022 16:00:00'' y ``28-03-2022 17:00:00'' se capturan aproximadamente 1800 lecturas de potencia, para considerar la facturación se obtiene la media aritmética del valor de potencia del conjunto y se almacena como único registro en la tabla de facturación con fecha y hora '28-03-2022 17:00:00'. posteriormente se procede a eliminar los 1800 registros temporales utilizados. Los consumos no facturados son los registros temporales. 


Se realizó un muestreo de 5 horas de funcionamiento continuo del ventilador y se obtuvieron los resultados que se muestran en la figura \ref{fig:dashboard-consumo}. El \emph{dashboard} del \emph{software} de monitoreo y control ofrece una vista resumida del subtotal y total a pagar así como un gráfico interactivo de potencia en función del costo de consumo por hora.

%%%%%%%%%%%%%%%%%%%%%%%%%%%%%%%%%%%%%%%%%%%%%%%%%%%
\begin{landscape} % esto es para rotar la pagina e imagen
\begin{figure}[htpb]
\centering 
\includegraphics[width=1.7\textwidth]{./Figures/test/consumo/actuador1.png}
\caption{Visualización del módulo con paso de corriente eléctrica.}
\label{fig:dashboard-v1}
\end{figure}
\end{landscape} %
%%%%%%%%%%%%%%%%%%%%%%%%%%%%%%%%%%%%%%%%%%%%%%%%%%%
%%%%%%%%%%%%%%%%%%%%%%%%%%%%%%%%%%%%%%%%%%%%%%%%%%%
\begin{landscape} % esto es para rotar la pagina e imagen
\begin{figure}[htpb]
\centering 
\includegraphics[width=1.7\textwidth]{./Figures/test/consumo/actuador2.png}
\caption{Visualización del módulo con bloqueo de la corriente eléctrica.}
\label{fig:dashboard-v2}
\end{figure}
\end{landscape} %
%%%%%%%%%%%%%%%%%%%%%%%%%%%%%%%%%%%%%%%%%%%%%%%%%%%
%%%%%%%%%%%%%%%%%%%%%%%%%%%%%%%%%%%%%%%%%%%%%%%%%%%
\begin{landscape} % esto es para rotar la pagina e imagen
\begin{figure}[htpb]
\centering 
\includegraphics[width=1.7\textwidth]{./Figures/test/consumo/consumo.png}
\caption{Dashboard de facturación del software de monitoreo y control.}
\label{fig:dashboard-consumo}
\end{figure}
\end{landscape} %
%%%%%%%%%%%%%%%%%%%%%%%%%%%%%%%%%%%%%%%%%%%%%%%%%%%

Los valores obtenidos en la figura \ref{fig:dashboard-consumo} fueron contrastados con los valores ideales para el funcionamiento de 5 horas de uso. Usando la ecuación \ref{eq:consumoform} surge el resultado que se muestra en la ecuación \ref{eq:ec}:

\begin{equation}
	\label{eq:ec}
	EC = \left( 0.15 \right) kW
\end{equation}


Para obtener el costo de consumo, se multiplica el costo de un kWh por el total de consumo. Se utilizó la última lectura de potencia entregada por el \emph{software} de monitoreo que es igual 32.55 W, tal como se muestra en la figura \ref{fig:dashboard-v1}. La tabla \ref{tab:tablacostos} muestra la comparación entre los valores obtenidos considerando el uso del ventilador de una hora al día, por 5 horas por día, por semana (25 h) y por mes (100 h). Se consideró el costo de S/0,5 nuevos soles (en moneda de Perú) por cada 1 kWh.

\begin{table}[h]
	\centering
	\caption[Comparativa de consumos y costos]{Comparativa de consumos y costos.}
	\begin{tabular}{c c c c c c c}    
		\toprule
		\textbf{Pi (kW)} 	 & \textbf{Ps (kW)}  & \textbf{T (h)} &\textbf{Ci (kW)} &\textbf{Cs (kW)} &\textbf{Coi (S/)} &\textbf{Cos (S/)}\\
		\midrule
		0,03 & 0,03255 & 1 & 0,03 & 0,03255 & 0,015 & 0,016275\\		
		0,03 & 0,03255 & 5 & 0,15 & 0,16275 & 0,075 & 0,081375 \\
		0,03 & 0,03255 & 25 & 0,75 & 0,81375 & 0,375 & 0,406875\\		
		0,03 & 0,03255 & 100 & 3 & 3,255 & 1,5 & 1,6275\\		
		
		\bottomrule
		\hline
	\end{tabular}
	\label{tab:tablacostos}
\end{table}

\vspace{0.1cm}
El significado de las abreviaturas de las columnas son:
\begin{itemize}
\item Pi: potencia ideal (según descripción técnica del ventilador).
\item Ps: potencia entregada por el \emph{software} de monitoreo y control.
\item T: tiempo.
\item Ci: consumo ideal o esperado.
\item Cs: consumo entregado por el \emph{software} de monitoreo y control.
\item Coi: costo ideal o esperado.
\item Cos: costo calculado por el \emph{software} de monitoreo y control.
\end{itemize}

\vspace{0.1cm}
De la columna Coi (valor esperado) y Cos (valor obtenido) es posible observar que los valores entregados por el \emph{software} son muy cercanos al esperado. Para obtener los errores relativos (Er) y errores absolutos (Ea) se utilizaron las ecuaciones  \ref{eq:ea} y \ref{eq:er} . La tabla \ref{tab:tablaerror} muestra los resultados de los errores obtenidos del muestreo.


%\begin{equation}
%	\label{eq:vp}
%	\overline{X} = \frac1n \cdot \sum_{i=0}^n X_i  
%\end{equation}

\begin{equation}
	\label{eq:ea}
	E_a = \left| V_r - V_a \right|
\end{equation}

\begin{equation}
	\label{eq:er}
	E_r = \left( \frac{E_a}{V_r} \right)
\end{equation}

%\vspace{1.0cm}
Siendo las variables:
\begin{itemize}
\item Ea: error absoluto. 
\item Er: error relativo.
\item Vr: valor real o esperado.
\item Va: valor aproximado o medido.
\end{itemize}

\vspace{0.5cm}
\begin{table}[h]
	\centering
	\caption[Error absoluto y relativo]{Error absoluto y relativo.}
	\begin{tabular}{c c c c c}    
		\toprule
		\textbf{T (h)} & \textbf{Coi (Vr)} &\textbf{Cos (Va)} &\textbf{Ea} &\textbf{Er}\\
		\midrule
		1 & 0,015 & 0,016275 & 0,001275 & 0,085 \\		
		5 & 0,075 & 0,081375 & 0,006375 & 0,085 \\
		25 & 0,375 & 0,406875 & 0,031875 & 0,085\\		
		100 & 1,5 & 1,6275 & 0,1 & 0,067\\		
		
		\bottomrule
		\hline
	\end{tabular}
	\label{tab:tablaerror}
\end{table}

\section{Pruebas del funcionamiento del módulo replicador}

Este módulo esta compuesto por un conjunto de subprocesos internos que se ejecutan de forma continua dentro del sistema operativo y cada subproceso muestra resultados de su salida por la terminal según su comportamiento. En consideración al fin comercial del prototipo, se ocultó la información sensible que usa el software para su funcionamiento.

El proceso de envío de valores a la nube esta dividido en dos subprocesos. El primero es el envío de registros mediante la API remota para replicar los registros que se envían a la base de datos (P6), el segundo es el envío de valores al broker remoto (P1). La figura \ref{fig:envia1} muestra la salida del subproceso P6 durante el llamado a la API y la figura \ref{fig:envia2} muestra la salida del subproceso P1 durante el envío al broker remoto.

\begin{figure}[htpb]
\centering 
\includegraphics[width=0.9\textwidth]{./Figures/test/replicador/enviaAPI.png}
\caption{Salida del proceso de replicación a la nube utilizando la API.}
\label{fig:envia1}
\end{figure}

\begin{figure}[htpb]
\centering 
\includegraphics[width=1.0\textwidth]{./Figures/test/replicador/enviabroker.png}
\caption{Salida del proceso de replicación al broker remoto.}
\label{fig:envia2}
\end{figure}

Se mencionó en los capítulos anteriores que el replicador está constantemente verificando si existen registros que no han sido replicados a la nube, de ser así automáticamente llama a los subprocesos P5 y P8 para enviar los registros detectados mediante la API. La figura \ref{fig:r2} muestra la salida del subproceso P5 al replicar registros usando la API.

%La figura \ref{fig:r1} muestra la salida del subproceso  P5 sin registros por enviar a la nube y l

%\begin{figure}[htpb]
%\centering 
%\includegraphics[width=0.8\textwidth]{./Figures/test/replicador/sinReplicar1.png}
%\caption{Salida del proceso cuando no existen registros pendientes por replicar.}
%\label{fig:r1}
%\end{figure}

\begin{figure}[htpb]
\centering 
\includegraphics[width=1.0\textwidth]{./Figures/test/replicador/sinReplicar2.png}
\caption{Salida del proceso de replicación a la nube de registros marcados como no replicado.}
\label{fig:r2}
\end{figure}

%\section{Pruebas del sistema desde acceso remoto}

%Para visualizar la comunicación entre el módulo replicador y el módulo principal local se muestra el grafo del funcionamiento del sistema de monitoreo y control. El software permite visualizar en tiempo real el comportamiento y el envío de mensajes entre canales como se muestra en la figura \ref{fig:graforemoto}. Los puntos de colores en la figura significan:

%\begin{itemize}
%\item Punto negro: mensaje enviado.
%\item Punto rojo: cliente remoto conectado al sistema IoT.
%\item Punto azul: cliente conectado al broker remoto en modo escucha.
%\item Punto naranja: mensaje de un cliente remoto conectado temporalmente para enviar sincronización de estados desde el remoto al modulo principal local. 
%\end{itemize}

%%%%%%%%%%%%%%%%%%%%%%%%%%%%%%%%%%%%%%%%%%%%%%%%%%%
%\begin{landscape} % esto es para rotar la pagina e imagen
%\begin{figure}[htpb]
%\centering 
%\includegraphics[width=1.6\textwidth]{./Figures/test/replicador/remoto.png}
%\caption{Grafo de comunicación MQTT funcionando desde el sistema réplica en la nube.}
%\label{fig:graforemoto}
%\end{figure}
%\end{landscape} 
%%%%%%%%%%%%%%%%%%%%%%%%%%%%%%%%%%%%%%%%%%%%%%%%%%%


\section{Pruebas del funcionamiento del sistema sin Internet}

El módulo replicador contiene un subproceso que se ejecuta de forma continua y verifica el estado del servicio de Internet. Al estar sin Internet en la red, el sistema detecta la falta de conectividad y muestra un mensaje en la parte superior del \emph{software} de monitoreo y control. La figura \ref{fig:inter1} muestra la salida por la terminal de los procesos del replicador al detectar la ausencia del servicio de Internet y la figura \ref{fig:inter3} muestran la alerta en el \emph{software}. 



%%%%%%%%%%%%%%%%%%%%%%%%%%%%%%%%%%
\begin{figure}[htpb]
\centering 
\includegraphics[width=1.0\textwidth]{./Figures/test/replicador/desconexion3.png}
\caption{Salida del proceso al detectar corte de Internet.}
\label{fig:inter1}
\end{figure}
\vspace{0.25cm}
%%%%%%%%%%%%%%%%%%%%%%%%%%%%%%
%\begin{figure}[htpb]
%\centering 
%\includegraphics[width=0.65\textwidth]{./Figures/test/replicador/desconexion1.png}
%\caption{Estado del software con Internet activo.}
%\label{fig:inter2}
%\end{figure}
%%%%%%%%%%%%%%%%%%%%%%%%%%%%%%%%%%%%%%
\begin{figure}[htpb]
\centering 
\includegraphics[width=1.0\textwidth]{./Figures/test/replicador/desconexion2.png}
\caption{Estado del \emph{software} sin Internet.}
\label{fig:inter3}
\end{figure}


\section{Comparativa del resultado con soluciones similares}

Una vez expuestos los resultados obtenidos para cada módulo, se presenta a continuación el análisis de los resultados comparativos entre los tres tipos de soluciones descritas en el capítulo 1. La tabla \ref{tab:tabla-resultado} permite determinar las diferencias significativas con el uso y el tipo de servidor contra cada una de las diferentes soluciones.

\begin{table}[h]
	\centering
	\caption[Comparativa de soluciones entre acceso y servidor]{Comparativa acceso y tipo de servidor.}
	\begin{tabular}{p{4cm} c c c c }    
		\toprule
		\textbf{Producto} & \textbf{Acceso} & \textbf{Servidor local}   & \textbf{Servidor remoto} \\
		\midrule
		Energy Manager & Local y remoto  & no & sí  \\		
		Iammeter	 & local y remoto & no & sí  \\
		Bee2energy	 & local y remoto	& no & sí  \\
		Cenergy IoT System (prototipo) & local y remoto & sí & sí \\
		\bottomrule
		\hline
	\end{tabular}
	\label{tab:tabla-resultado}
\end{table}

%%%%%%%%%%%%%%%%%%%%%%

La tabla \ref{tab:tabla-resultado2} permite determinar las diferencias respecto al uso de protocolos y el tipo de módulos que usa cada solución. 


\begin{table}[h]
	\centering
	\caption[Comparativa de soluciones entre protocolo y hardware]{Comparativa protocolo y tipos de hardware.}
	\begin{tabular}{p{4cm} p{3cm} p{2cm} p{2cm}}    
		\toprule
		\textbf{Producto} 	 & \textbf{Protocolos}  & \textbf{Sensores} & \textbf{Actuadores}  \\
		\midrule
		Energy Manager & Modbus, M-Bus y TCP/IP & propietario & propietario \\		
		Iammeter	 & MQTT y TCP/IP	& propietario / comercial & propietario / comercial   \\
		Bee2energy	 & múltiples protocolos IoT		& propietario / comercial & propietario / comercial \\
		Cenergy IoT	System (prototipo) & MQTT y TCP/IP	& propietario & propietario  \\
		\bottomrule
		\hline
	\end{tabular}
	\label{tab:tabla-resultado2}
\end{table}
 
% Chapter Template

\chapter{Conclusiones} % Main chapter title

\label{Chapter5} % Change X to a consecutive number; for referencing this chapter elsewhere, use \ref{ChapterX}

En este capítulo se detallan las conclusiones relacionadas al alcance de los objetivos que se plantearon al inicio del trabajo. Además, se analizan las características de \emph{software} y \emph{hardware} del prototipo desarrollado, el cumplimiento de la planificación y los próximos pasos a seguir para mejorarlo y convertirlo en un producto comercial.


%----------------------------------------------------------------------------------------

%----------------------------------------------------------------------------------------
%	SECTION 1
%----------------------------------------------------------------------------------------

\section{Conclusiones generales }

Este trabajo logró desarrollar de forma exitosa un sistema IoT para monitoreo y control de viviendas y edificios. Particularmente se implementó el monitoreo, supervisión y control de temperatura, humedad y medición de características de la red eléctrica del edificio. Se verificó el cumplimiento de los requerimientos más importantes, quedando aún por validar algunos menos relevantes y que pueden ser implementados en trabajos futuros con un cronograma con mayor margen de tiempo.




%Los módulos que requirieron de \emph{hardware} y \emph{firmware} para cumplir su función dentro del sistema IoT fueron desarrollados considerando los requerimientos funcionales y el tiempo marcado en el cronograma como principales objetivos a cumplir. 

Se realizaron pruebas principales de usabilidad, funcionalidad, acceso de red y seguridad web mínimos y necesarios para poder comprobar su funcionamiento en un ambiente real. 

%Todos los componentes de software desarrollados para el sistema IoT fueron creados y testeados dentro de un entorno operativo Windows 10, GNU/Linux Elementary OS y RasberryPi OS. 

%Para mayor información y seguimiento de futuros desarrollos funcionales, se puede acceder a la web oficial: www.cenergy.icfnet.org

Durante el desarrollo de este trabajo se aplicaron conocimientos adquiridos a lo largo de la especialización. A continuación, se detallan las que tuvieron mayor relevancia:


\begin{itemize}
\item Gestión de proyectos: se elaboró un plan de proyecto, pudiendo contar desde el comienzo con una planificación clara del trabajo a realizar.

\item Protocolos de Internet: se aprendió sobre las capas de red, protocolos de red y configuración para redes LAN y WLAN. 

\item Arquitectura de protocolos: se aplicaron los conocimientos sobre el protocolo MQTT para el esquema de comunicación de sistemas IoT.

\item Desarrollo de aplicaciones web: se utilizaron buenas prácticas de programación y patrones de desarrollo, especialmente apropiadas para aplicaciones web. 

\item Infraestructura para la implementación de sistemas: se usaron conceptos vistos sobre el diseño, gestión e interacción de una aplicación web con una base de datos.

\item Ciberseguridad en IoT: permitió conocer los posibles escenarios críticos e identificar vulnerabilidades que deben ser tomadas en cuenta para el desarrollo de software seguro.

\end{itemize}

Por lo tanto, se concluye que los objetivos planteados al inicio del trabajo han sido alcanzados satisfactoriamente y se han obtenido y reforzado conocimientos valiosos para la formación profesional del autor.


%----------------------------------------------------------------------------------------
%	SECTION 2
%----------------------------------------------------------------------------------------
\section{Próximos pasos}

Para dar continuidad al esfuerzo realizado hasta el momento y poder obtener un producto comercialmente atractivo surgen los siguientes puntos: 

\begin{itemize}

\item Rediseñar cada módulo físico y unificar los componentes electrónicos internos en una placa de circuito impreso o PCB (por las siglas en ingles \emph{printed circuit board}) más pequeña, considerando estándares de fabricación de placas electrónicas para uso comercial.

%Integrar una unidad de almacenamiento interno en el módulo actuador para guardar el ultimo estado valido antes de ser apagado y de esta manera acelerar la sincronización del modulo al ser encendido nuevamente.

\item Implementar nuevas funciones de ciberseguridad web para el software de monitoreo local y remoto considerando los ataques cibernéticos más comunes en dicho entorno.

\item Desarrollar la autenticación por token vía SMS para la validación de acceso al software principal de monitoreo y control, para garantizar una capa de seguridad web adicional al sistema actual.

\item Desarrollar una aplicación móvil para entornos Android e IOS para facilitar el acceso y agregado de módulos al sistema IoT y garantizar un servicio más amigable al usuario.

\item Implementar mecanismos de cifrado de unidades internas de la tarjeta microSD del módulo principal donde se almacena el software de monitoreo, los procesos internos de red y la base de datos local, para evitar la fácil manipulación de información confidencial del sistema.
\end{itemize}
 

%----------------------------------------------------------------------------------------
%	CONTENIDO DE LA MEMORIA  - APÉNDICES
%----------------------------------------------------------------------------------------

\appendix % indicativo para indicarle a LaTeX los siguientes "capítulos" son apéndices

% Incluir los apéndices de la memoria como archivos separadas desde la carpeta Appendices
% Descomentar las líneas a medida que se escriben los apéndices

%\include{Appendices/AppendixA}
%\include{Appendices/AppendixB}
%\include{Appendices/AppendixC}

%----------------------------------------------------------------------------------------
%	BIBLIOGRAPHY
%----------------------------------------------------------------------------------------

\Urlmuskip=0mu plus 1mu\relax
\raggedright
\printbibliography[heading=bibintoc]



%----------------------------------------------------------------------------------------

\end{document}  
