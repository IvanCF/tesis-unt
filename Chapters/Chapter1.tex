% Chapter 1

\chapter{Introducción general} % Main chapter title

\label{Chapter1} % For referencing the chapter elsewhere, use \ref{Chapter1} 
\label{IntroGeneral}
La tecnología de automatización orientada a viviendas y lugares de trabajo está tomando cada vez mayor relevancia en nuestras vidas, aunque existe una gran variedad de soluciones que ofrecen dotar de tecnología a ciertos elementos en un ambiente doméstico, su principal desventaja es que casi siempre la recolección de datos se realiza con sensores que dependen de una conexión a Internet para enviar sus valores a un servidor central para su gestión. Esta forma de trabajo permite la facilidad de uso pero con la desventaja de que la gestión funciona solo si existe conexión a Internet, además los sensores más comerciales del mercado actual están destinados a un uso independiente con su propia aplicación móvil, complicando la necesidad de tener una única solución centralizada ante el uso de múltiples sensores.

El presente proyecto se destaca especialmente por centralizar y unificar resultados de la red de sensores en un sistema web principal de monitoreo y control sin depender de la conexión a Internet para su funcionamiento. Para lograr esta tarea se trabajará con la recolección de datos de sensores ubicados en distintos puntos de estudio de una vivienda o ambiente conectados a una red local dedicada. Cada una de las lecturas de los sensores serán enviadas a una unidad central local mediante el protocolo MQTT por vía inalámbrica, dejando la dependencia del Internet solo para los accesos remotos.

El módulo central local contará con la función para replicar los valores recolectados hacia un servicio en la nube, tarea que se realizará mientras tenga conexión a Internet.

Las funciones principales a implementar están orientados a confort (control de ventiladores y luces) y consumo energético de electrodomésticos.  

El sistema BMS a desarrollar contará de dos formas de acceso: la primera es vía red local y la segunda vía Internet desde cualquier dispositivo mediante un navegador.
%----------------------------------------------------------------------------------------

% Define some commands to keep the formatting separated from the content 
\newcommand{\keyword}[1]{\textbf{#1}}
\newcommand{\tabhead}[1]{\textbf{#1}}
\newcommand{\code}[1]{\texttt{#1}}
\newcommand{\file}[1]{\texttt{\bfseries#1}}
\newcommand{\option}[1]{\texttt{\itshape#1}}
\newcommand{\grados}{$^{\circ}$}

%----------------------------------------------------------------------------------------

%\section{Introducción}

%----------------------------------------------------------------------------------------

%\section{Conceptos generales}

%En esta sección se describen aspectos esenciales para poder conocer y entender las tecnologías %y servicios usados en el trabajo.
\vspace{5.0cm}

\section{Justificación de la investigación}

Las necesidades de crear sistemas autómatas que permitan el control, seguridad, ahorro de energía y comunicación, son prestaciones que aportan valor añadido a la gestión técnica en los proyectos de edificaciones para los sectores público y privado creando entornos y ambientes que facilitan la habitabilidad y uso de una infraestructura moderna que aprovecha y hace uso de los diversos avances tecnológicos.

El desarrollo sostenible del país, ha permitido que en las últimas décadas los sectores productivos, tales como la industria, la construcción, la minería, entre otros, experimenten un auge y crecimiento de proyectos de inversión pública y privada. Para lo cual se genera la necesidad de contar con profesionales altamente capaces de aprovechar y desarrollar los diversos medios, elementos y avances tecnológicos en beneficio de la sociedad y los sectores productivos del país. \citep{ARTICLE:2}.

Por lo mencionado en los párrafos anteriores nuestra labor como profesionales con bases tecnológicas y científicas es crear los medios que permita acercar la tecnología a las personas, brindándoles un abanico de soluciones cada vez más amigables e informativas para que puedan conocer y usar. Por tanto, este proyecto busca ser una alternativa para aquellas personas que buscan una solución de automatización inteligente para sus viviendas u oficinas, así como brindarles la capacidad de interconexión de sus dispositivos por medio de nuestra plataforma.

\section{Formulación del problema}

¿Cómo mejorar la gestión y control del consumo eléctrico mediante un sistema BMS basado en los principios de IoT?

\section{Hipótesis}

El desarrollo de un sistema BMS - IoT para automatizar la gestión eléctrica permitirá reducir el error humano en la toma de muestras y cálculo del consumo eléctrico mensual dentro de un edificio habitacional, de oficinas u otro entorno. Reduciendo el tiempo de muestreo que esta tarea conlleva y aumentando la disponibilidad de los datos de consumo para su acceso en tiempo real.

\section{Objetivos}

El propósito de este trabajo es diseñar y desarrollar un sistema prototipo operativo funcional capaz de controlar y monitorear el consumo eléctrico mediante el protocolo MQTT y para lograrlo se plantea los siguientes objetivos.

\subsection{Generales}
\begin{itemize}
\item Diseñar y desarrollar un sistema BMS - IoT para el la gestión y control 
automatizado del consumo eléctrico utilizando el protocolo MQTT.
\end{itemize}
\subsection{Específicos}

\begin{itemize}
\item Diseñar una red IoT para un sistema BMS considerando las variables existentes en los edificios habitacionales.
\item Diseñar e Implementar la interfaz de comunicación y toma de datos de los sensores de forma simultanea mediante comunicación inalámbrica con el protocolo MQTT.
\item Diseñar e Implementar un módulo integrador - replicador de comunicación entre la red interna y la comunicación en la nube.
\item Diseñar e Implementar los módulos electrónicos con sensores y actuadores para medir el consumo eléctrico y temperatura ambiente.
\item Diseñar e implementar un software web responsivo a medida para el sistema BMS - IoT.
\end{itemize}

%-----------------------------------------------------------------------



%----------------------------------------------------------------------------------------
%\section{Alcance}


%\subsection{Alcance}

%El desarrollo de trabajo incluye:

%\begin{itemize}
%\item Diseño e implementación de un módulo principal local.
%\item Diseño e implementación del módulo replicador de datos local-nube.
%\item Diseño e implementación del módulo actuador y de consumo eléctrico.
%\item Diseño e implementación módulo de control de temperatura.
%\item Diseño e implementación de un \emph{software} web BMS a medida.
%\end{itemize}

%\subsection{Objetivos}
%\begin{itemize}
%\item Diseñar y desarrollar un sistema IoT para medir el consumo eléctrico.
%\item Diseñar y desarrollar un sistema que no dependa de una conexión a Internet para su funcionamiento.
%\item Diseñar y desarrollar un \emph{software} a medida para la gestión de control y monitoreo de una vivienda o edificio.
%\item Diseñar e implementar módulos con comunicación Wi-Fi para el control y monitoreo de energía eléctrica.
%\end{itemize}

%----------------------------------------------------------------------------------------
\let\cleardoublepage\clearpage % para eliminar pagina en blanco siguiente