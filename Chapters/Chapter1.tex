% Chapter 1

\chapter{Introducción general} % Main chapter title

\label{Chapter1} % For referencing the chapter elsewhere, use \ref{Chapter1} 
\label{IntroGeneral}
En este capítulo se presenta una introducción técnica a Internet de las cosas, se describe la motivación, se presentan otros sistemas comerciales del mercado y se detallan los objetivos y alcance del presente trabajo.
%----------------------------------------------------------------------------------------

% Define some commands to keep the formatting separated from the content 
\newcommand{\keyword}[1]{\textbf{#1}}
\newcommand{\tabhead}[1]{\textbf{#1}}
\newcommand{\code}[1]{\texttt{#1}}
\newcommand{\file}[1]{\texttt{\bfseries#1}}
\newcommand{\option}[1]{\texttt{\itshape#1}}
\newcommand{\grados}{$^{\circ}$}

%----------------------------------------------------------------------------------------

%\section{Introducción}

%----------------------------------------------------------------------------------------

%\section{Conceptos generales}

%En esta sección se describen aspectos esenciales para poder conocer y entender las tecnologías %y servicios usados en el trabajo.

\section{Justificación de la investigación}

Las necesidades de crear sistemas autómatas que permitan el control, seguridad, ahorro de energía y comunicación, son prestaciones que aportan valor añadido a la gestión técnica en los proyectos de edificaciones para los sectores público y privado creando entornos y ambientes que facilitan la habitabilidad y uso de una infraestructura moderna que aprovecha y hace uso de los diversos avances tecnológicos.

El desarrollo sostenible del país, ha permitido que en las últimas décadas los sectores productivos, tales como la industria, la construcción, la minería, entre otros, experimenten un auge y crecimiento de proyectos de inversión pública y privada. Para lo cual se genera la necesidad de contar con profesionales altamente capaces de aprovechar y desarrollar los diversos medios, elementos y avances tecnológicos en beneficio de la sociedad y los sectores productivos del país. [2].....

Por lo mencionado en los párrafos anteriores nuestra labor como profesionales con bases tecnológicas y científicas es crear los medios que permita acercar la tecnología a las personas, brindándoles un abanico de soluciones cada vez más amigables e informativas para que puedan conocer y usar. Por tanto, este proyecto busca ser una alternativa para aquellas personas que buscan una solución de automatización inteligente para sus viviendas u oficinas, así como brindarles la capacidad de interconexión de sus dispositivos por medio de nuestra plataforma.

\section{Formulación del problema}

¿Cómo mejorar la gestión y control del consumo eléctrico mediante un sistema BMS basado en los principios de IoT?

\section{Hipótesis}

texto ....

\section{Objetivos}


\subsection{Objetivos generales}

Diseñar y desarrollar un sistema BMS - IoT para el la gestión y control 
automatizado del consumo eléctrico utilizando el protocolo MQTT.

\subsection{Objetivos específicos}

texto ....


%-----------------------------------------------------------------------



%----------------------------------------------------------------------------------------
\section{Propósito y alcance}



%\subsection{Propósito}

El propósito de este trabajo es diseñar y desarrollar un sistema prototipo operativo funcional capaz de controlar y monitorear el consumo eléctrico de los  edificios mediante el protocolo MQTT para brindar una gestión inteligente respecto a confort y consumo energético.

%\subsection{Alcance}

El alcance del proyecto incluye:
\begin{itemize}
\item Diseño y desarrollo de un módulo principal local.
\item Diseño y desarrollo del módulo replicador de datos local-nube.
\item Diseño y desarrollo del módulo actuador y de consumo eléctrico.
\item Diseño y desarrollo del módulo de control de temperatura.
\item Diseño y desarrollo de un \emph{software} web BMS a medida.
\end{itemize}

%\subsection{Objetivos}
%\begin{itemize}
%\item Diseñar y desarrollar un sistema IoT para medir el consumo eléctrico.
%\item Diseñar y desarrollar un sistema que no dependa de una conexión a Internet para su funcionamiento.
%\item Diseñar y desarrollar un \emph{software} a medida para la gestión de control y monitoreo de una vivienda o edificio.
%\item Diseñar e implementar módulos con comunicación Wi-Fi para el control y monitoreo de energía eléctrica.
%\end{itemize}

%----------------------------------------------------------------------------------------
\let\cleardoublepage\clearpage % para eliminar pagina en blanco siguiente