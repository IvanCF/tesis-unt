% Chapter Template

\chapter{Consideraciones finales} % Main chapter title

\label{Chapter5} % Change X to a consecutive number; for referencing this chapter elsewhere, use \ref{ChapterX}

En este capítulo se detallan las conclusiones relacionadas al alcance de los objetivos que se plantearon al inicio del trabajo. Además, se analizan las características de \emph{software} y \emph{hardware} del prototipo desarrollado, el cumplimiento de la planificación y los próximos pasos a seguir para mejorarlo y convertirlo en un producto comercial.


%----------------------------------------------------------------------------------------

%----------------------------------------------------------------------------------------
%	SECTION 1
%----------------------------------------------------------------------------------------

\section{Conclusiones generales }

Este trabajo logró desarrollar de forma exitosa un sistema IoT para monitoreo y control de viviendas y edificios. Particularmente se implementó el monitoreo, supervisión y control de temperatura, humedad y medición de características de la red eléctrica del edificio. Se verificó el cumplimiento de los requerimientos más importantes, quedando aún por validar algunos menos relevantes y que pueden ser implementados en trabajos futuros con un cronograma con mayor margen de tiempo.




%Los módulos que requirieron de \emph{hardware} y \emph{firmware} para cumplir su función dentro del sistema IoT fueron desarrollados considerando los requerimientos funcionales y el tiempo marcado en el cronograma como principales objetivos a cumplir. 

Se realizaron pruebas principales de usabilidad, funcionalidad, acceso de red y seguridad web mínimos y necesarios para poder comprobar su funcionamiento en un ambiente real. 

%Todos los componentes de software desarrollados para el sistema IoT fueron creados y testeados dentro de un entorno operativo Windows 10, GNU/Linux Elementary OS y RasberryPi OS. 

%Para mayor información y seguimiento de futuros desarrollos funcionales, se puede acceder a la web oficial: www.cenergy.icfnet.org

Durante el desarrollo de este trabajo se aplicaron conocimientos adquiridos a lo largo de la especialización. A continuación, se detallan las que tuvieron mayor relevancia:


\begin{itemize}
\item Gestión de proyectos: se elaboró un plan de proyecto, pudiendo contar desde el comienzo con una planificación clara del trabajo a realizar.

\item Protocolos de Internet: se aprendió sobre las capas de red, protocolos de red y configuración para redes LAN y WLAN. 

\item Arquitectura de protocolos: se aplicaron los conocimientos sobre el protocolo MQTT para el esquema de comunicación de sistemas IoT.

\item Desarrollo de aplicaciones web: se utilizaron buenas prácticas de programación y patrones de desarrollo, especialmente apropiadas para aplicaciones web. 

\item Infraestructura para la implementación de sistemas: se usaron conceptos vistos sobre el diseño, gestión e interacción de una aplicación web con una base de datos.

\item Ciberseguridad en IoT: permitió conocer los posibles escenarios críticos e identificar vulnerabilidades que deben ser tomadas en cuenta para el desarrollo de software seguro.

\end{itemize}

Por lo tanto, se concluye que los objetivos planteados al inicio del trabajo han sido alcanzados satisfactoriamente y se han obtenido y reforzado conocimientos valiosos para la formación profesional del autor.


%----------------------------------------------------------------------------------------
%	SECTION 2
%----------------------------------------------------------------------------------------
\section{Trabajos futuros}

Para dar continuidad al esfuerzo realizado hasta el momento y poder obtener un producto comercialmente atractivo surgen los siguientes puntos: 

\begin{itemize}

\item Rediseñar cada módulo físico y unificar los componentes electrónicos internos en una placa de circuito impreso o PCB (por las siglas en ingles \emph{printed circuit board}) más pequeña, considerando estándares de fabricación de placas electrónicas para uso comercial.

%Integrar una unidad de almacenamiento interno en el módulo actuador para guardar el ultimo estado valido antes de ser apagado y de esta manera acelerar la sincronización del modulo al ser encendido nuevamente.

\item Implementar nuevas funciones de ciberseguridad web para el software de monitoreo local y remoto considerando los ataques cibernéticos más comunes en dicho entorno.

\item Desarrollar la autenticación por token vía SMS para la validación de acceso al software principal de monitoreo y control, para garantizar una capa de seguridad web adicional al sistema actual.

\item Desarrollar una aplicación móvil para entornos Android e IOS para facilitar el acceso y agregado de módulos al sistema IoT y garantizar un servicio más amigable al usuario.

\item Implementar mecanismos de cifrado de unidades internas de la tarjeta microSD del módulo principal donde se almacena el software de monitoreo, los procesos internos de red y la base de datos local, para evitar la fácil manipulación de información confidencial del sistema.
\end{itemize}
